C:/Programacao/_MinhasBibliotecas/Jupyter2Latex/latex-template/article-base.tex

\author{Eduardo Pagnussat Titello}
\title{PEC00144 - Métodos Experimentais em Engenharia Civil}
\subtitle{Trabalho 3B - Prévia do trabalho final}
\date{Janeiro de 2021}


\begin{document}
	
\maketitle



\section{Introdução}
A frequência de vibração livre de um sistema de um único grau de liberdade é dada por: 

\begin{equation}
	f_n = \frac{1}{2\pi} \sqrt{\frac{k}{m}}
	\label{eq:fn}
\end{equation}

onde $k$ representa a rigidez do sistema e $m$ sua massa.

Embora sistemas estruturais com múltiplos graus de liberdade, como edifícios, requisitem análises de maior complexidade, o vínculo entre as frequências de vibração, a rigidez do sistema e sua massa é mantido. Dessa forma quanto menor a massa do sistema maior sua frequência de vibração. 


\section{Experimento proposto}



\section{Introdução}
A análise dinâmica de edifícios de múltiplos pavimentos pode ser realizada matematicamente com modelos de diferentes graus de refinamento. Esses modelos costumam ter suas massas concentradas nos pavimentos e podem ser do tipo \textit{shear building} ou tridimensionais (Soriano, 2014).

O modelo \textit{shear building}, clássico e de grande simplicidade, supõe pisos indeformáveis e colunas inextensíveis. Esse modelo equivale à uma coluna de trechos de rigidezes iguais à soma das rigidezes à flexão dos pilares de cada pavimento da edificação (Soriano, 2014). Na figura \ref{fig:shearb} são apresentados um modelo de edifício com duas colunas de pilares e sua representação como \textit{shear building}.


\begin{figure}
	\centering
	\caption{Modelo de edifício e sua representação como \textit{shear building}.\\ \small{(Adaptado de Soriano, 2014)}}
	\includegraphics[scale=0.4]{../Trabalho3B/Images/ShearBuilding}
	\label{fig:shearb}
\end{figure}



\section{Experimento proposto}

Embora o modelo matemático do \textit{shear building} seja formado por uma única coluna, conforme representação à direita na figura \ref{fig:shearb}, para fins de experimentação o uso de tal modelo pode ser inviável. Como exemplo tem-se a construção de modelos com amortecedores, que requerem uma área para sua instalação, 

A construção de modelos reduzidos com amortecedores, por exemplo, requer um modelo bidimensional, 

Dada a necessidade de espaço e estabilidade para instalação de sensores, massas para ajuste de escala, 



\section{Metodologia}
Aaaa


\section{Materiais}
Aaaa


\section{Fatores de escala}
Aaaa


\section{Resultados esperados}
Aaaa


\section{Referências Bibliográficas}
Aaaaa
    
\end{document}


