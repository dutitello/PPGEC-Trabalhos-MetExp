C:/Programacao/_MinhasBibliotecas/Jupyter2Latex/latex-template/article-base.tex

\author{Eduardo Pagnussat Titello}
\title{PEC00144 - Métodos Experimentais em Engenharia Civil}
\subtitle{Trabalho 5}
\date{Fevereiro de 2021}

\begin{document}
	
	\maketitle

Este trabalho tem por objetivo: fazer uma aquisição com o MPU6050 preso
numa régua em balanço e apresentar resultados no tempo e
na frequência. Determinar a frequência natural no modo fundamental da
régua.

    \begin{tcolorbox}[breakable, size=fbox, boxrule=1pt, pad at break*=1mm,colback=cellbackground, colframe=cellborder]
\prompt{In}{incolor}{1}{\boxspacing}
\begin{Verbatim}[commandchars=\\\{\}]
\PY{c+c1}{\PYZsh{} Importando e configurando módulos}
\PY{k+kn}{import} \PY{n+nn}{numpy} \PY{k}{as} \PY{n+nn}{np}
\PY{k+kn}{import} \PY{n+nn}{pandas} \PY{k}{as} \PY{n+nn}{pd} 
\PY{k+kn}{import} \PY{n+nn}{matplotlib}\PY{n+nn}{.}\PY{n+nn}{pyplot} \PY{k}{as} \PY{n+nn}{plt}
\PY{o}{\PYZpc{}}\PY{k}{config} InlineBackend.figure\PYZus{}format = \PYZsq{}svg\PYZsq{} \PYZsh{} Muda backend do jupyter para SVG ;)
\PY{k+kn}{import} \PY{n+nn}{jupyter2latex} \PY{k}{as} \PY{n+nn}{j2l} \PY{c+c1}{\PYZsh{} Uma maneira que encontrei para tabelas ficarem ok (github.com/dutitello/Jupyter2Latex)}
\PY{k+kn}{from} \PY{n+nn}{MRPy} \PY{k+kn}{import} \PY{n}{MRPy}
\PY{k+kn}{import} \PY{n+nn}{serial}
\PY{k+kn}{import} \PY{n+nn}{time}
\end{Verbatim}
\end{tcolorbox}

    \hypertarget{configurauxe7uxe3o-do-experimento}{%
\section{Configuração do
experimento}\label{configurauxe7uxe3o-do-experimento}}

No experimento é utilizada uma régua de alumínio com 53 cm de
comprimento e 2,5 cm de largura, onde o comprimento em balanço é 50 cm.
A configuração é apresentada na figura a seguir:

\begin{figure}
\centering
\includegraphics[width=9cm]{../Trabalho5/imagens/conf1}
\end{figure}

Os eixos do acelerômetro são posicionados conforme apresentado na figura a seguir:

\begin{figure}
\centering
\includegraphics[width=9cm]{../Trabalho5/imagens/PosEixos}
\end{figure}

    \hypertarget{leitura-e-processamento-das-acelerauxe7uxf5es}{%
\section{Leitura e processamento das
acelerações}\label{leitura-e-processamento-das-acelerauxe7uxf5es}}

Para leitura e processamento das acelerações são empregados os códigos
desenvolvidos por Rocha (2020), onde são introduzidas pequenas
alterações. Essencialmente os códigos são transformados em uma única
função onde os dados são lidos, reescalados e reamostrados, fixando o
\emph{time step}.

    \begin{tcolorbox}[breakable, size=fbox, boxrule=1pt, pad at break*=1mm,colback=cellbackground, colframe=cellborder]
\prompt{In}{incolor}{45}{\boxspacing}
\begin{Verbatim}[commandchars=\\\{\}]
\PY{k}{def} \PY{n+nf}{Acelr}\PY{p}{(}\PY{n}{nlines}\PY{p}{,} \PY{n}{port}\PY{o}{=}\PY{l+s+s1}{\PYZsq{}}\PY{l+s+s1}{COM3}\PY{l+s+s1}{\PYZsq{}}\PY{p}{,} \PY{n}{baud}\PY{o}{=}\PY{l+m+mi}{115200}\PY{p}{)}\PY{p}{:}
    \PY{c+c1}{\PYZsh{} Inicia conexão}
    \PY{n}{Ardn}  \PY{o}{=}  \PY{n}{serial}\PY{o}{.}\PY{n}{Serial}\PY{p}{(}\PY{n}{port}\PY{p}{,} \PY{n}{baud}\PY{p}{,} \PY{n}{timeout}\PY{o}{=}\PY{l+m+mi}{1}\PY{p}{)}

    \PY{c+c1}{\PYZsh{} Função que faz leitura }
    \PY{k}{def} \PY{n+nf}{ReadSerial}\PY{p}{(}\PY{n}{nchar}\PY{p}{,} \PY{n}{nvar}\PY{p}{,} \PY{n}{nlines}\PY{p}{)}\PY{p}{:}
        \PY{n}{Ardn}\PY{o}{.}\PY{n}{write}\PY{p}{(}\PY{n+nb}{str}\PY{p}{(}\PY{n}{nlines}\PY{p}{)}\PY{o}{.}\PY{n}{encode}\PY{p}{(}\PY{p}{)}\PY{p}{)}
        \PY{n}{data} \PY{o}{=} \PY{n}{np}\PY{o}{.}\PY{n}{zeros}\PY{p}{(}\PY{p}{(}\PY{n}{nlines}\PY{p}{,}\PY{n}{nvar}\PY{p}{)}\PY{p}{)}
        \PY{k}{for} \PY{n}{k} \PY{o+ow}{in} \PY{n+nb}{range}\PY{p}{(}\PY{n}{nlines}\PY{p}{)}\PY{p}{:}
            \PY{n}{wait} \PY{o}{=} \PY{k+kc}{True}
            \PY{k}{while}\PY{p}{(}\PY{n}{wait}\PY{p}{)}\PY{p}{:}
                \PY{k}{if} \PY{p}{(}\PY{n}{Ardn}\PY{o}{.}\PY{n}{inWaiting}\PY{p}{(}\PY{p}{)} \PY{o}{\PYZgt{}}\PY{o}{=} \PY{n}{nchar}\PY{p}{)}\PY{p}{:}
                    \PY{n}{wait} \PY{o}{=} \PY{k+kc}{False}
                    \PY{n}{bdat} \PY{o}{=} \PY{n}{Ardn}\PY{o}{.}\PY{n}{readline}\PY{p}{(}\PY{p}{)} 
                    \PY{n}{sdat} \PY{o}{=} \PY{n}{bdat}\PY{o}{.}\PY{n}{decode}\PY{p}{(}\PY{p}{)}
                    \PY{n}{sdat} \PY{o}{=} \PY{n}{sdat}\PY{o}{.}\PY{n}{replace}\PY{p}{(}\PY{l+s+s1}{\PYZsq{}}\PY{l+s+se}{\PYZbs{}n}\PY{l+s+s1}{\PYZsq{}}\PY{p}{,}\PY{l+s+s1}{\PYZsq{}}\PY{l+s+s1}{ }\PY{l+s+s1}{\PYZsq{}}\PY{p}{)}\PY{o}{.}\PY{n}{split}\PY{p}{(}\PY{p}{)}
                    \PY{n}{data}\PY{p}{[}\PY{n}{k}\PY{p}{,} \PY{p}{:}\PY{p}{]} \PY{o}{=} \PY{n}{np}\PY{o}{.}\PY{n}{array}\PY{p}{(}\PY{n}{sdat}\PY{p}{[}\PY{l+m+mi}{0}\PY{p}{:}\PY{n}{nvar}\PY{p}{]}\PY{p}{,} \PY{n}{dtype}\PY{o}{=}\PY{l+s+s1}{\PYZsq{}}\PY{l+s+s1}{int}\PY{l+s+s1}{\PYZsq{}}\PY{p}{)}
        \PY{k}{return} \PY{n}{data}

    \PY{c+c1}{\PYZsh{} Precisa deixar tipo 1 segundo para realizar conexão.}
    \PY{c+c1}{\PYZsh{} No caso aumentei o tempo para facilitar na hora da excitação. }
    \PY{c+c1}{\PYZsh{} OBS: A luz TX do arduino acende quando a conexão está ligada!}
    \PY{n+nb}{print}\PY{p}{(}\PY{l+s+s2}{\PYZdq{}}\PY{l+s+s2}{Wait for it...}\PY{l+s+s2}{\PYZdq{}}\PY{p}{)}
    \PY{n}{time}\PY{o}{.}\PY{n}{sleep}\PY{p}{(}\PY{l+m+mi}{3}\PY{p}{)}
    \PY{n+nb}{print}\PY{p}{(}\PY{l+s+s2}{\PYZdq{}}\PY{l+s+s2}{Reading, go!}\PY{l+s+s2}{\PYZdq{}}\PY{p}{)}

    \PY{k}{try}\PY{p}{:}
        \PY{n}{data} \PY{o}{=} \PY{n}{ReadSerial}\PY{p}{(}\PY{l+m+mi}{30}\PY{p}{,} \PY{l+m+mi}{4}\PY{p}{,} \PY{n}{nlines}\PY{p}{)}
        \PY{n}{t}    \PY{o}{=} \PY{n}{data}\PY{p}{[}\PY{p}{:}\PY{p}{,}\PY{l+m+mi}{0} \PY{p}{]}
        \PY{n}{acc}  \PY{o}{=} \PY{n}{data}\PY{p}{[}\PY{p}{:}\PY{p}{,}\PY{l+m+mi}{1}\PY{p}{:}\PY{p}{]}

        \PY{n}{Ardn}\PY{o}{.}\PY{n}{close}\PY{p}{(}\PY{p}{)}
        \PY{n+nb}{print}\PY{p}{(}\PY{l+s+s1}{\PYZsq{}}\PY{l+s+s1}{Acquisition ok!}\PY{l+s+s1}{\PYZsq{}}\PY{p}{)}
        
        \PY{c+c1}{\PYZsh{} Ajusta escala dos dados}
        \PY{n}{ti}  \PY{o}{=} \PY{p}{(}\PY{n}{t} \PY{o}{\PYZhy{}} \PY{n}{t}\PY{p}{[}\PY{l+m+mi}{0}\PY{p}{]}\PY{p}{)}\PY{o}{/}\PY{l+m+mi}{1000}
        \PY{n}{a}   \PY{o}{=}  \PY{l+m+mi}{2}\PY{o}{*}\PY{l+m+mf}{9.81}\PY{o}{*}\PY{n}{acc}\PY{o}{/}\PY{l+m+mi}{2}\PY{o}{*}\PY{o}{*}\PY{l+m+mi}{15}
        \PY{c+c1}{\PYZsh{} Fixa tamaho do timestep }
        \PY{n}{data} \PY{o}{=} \PY{n}{MRPy}\PY{o}{.}\PY{n}{resampling}\PY{p}{(}\PY{n}{ti}\PY{p}{,} \PY{n}{a}\PY{p}{)}
        \PY{n+nb}{print}\PY{p}{(}\PY{l+s+s1}{\PYZsq{}}\PY{l+s+s1}{Average sampling rate is }\PY{l+s+si}{\PYZob{}0:5.1f\PYZcb{}}\PY{l+s+s1}{Hz.}\PY{l+s+s1}{\PYZsq{}}\PY{o}{.}\PY{n}{format}\PY{p}{(}\PY{n}{data}\PY{o}{.}\PY{n}{fs}\PY{p}{)}\PY{p}{)}
        \PY{c+c1}{\PYZsh{} Entrega dados processados}
        \PY{k}{return} \PY{n}{data}
        
    \PY{k}{except}\PY{p}{:}
        \PY{n}{Ardn}\PY{o}{.}\PY{n}{close}\PY{p}{(}\PY{p}{)}
        \PY{n}{sys}\PY{o}{.}\PY{n}{exit}\PY{p}{(}\PY{l+s+s1}{\PYZsq{}}\PY{l+s+s1}{Acquisition failure!}\PY{l+s+s1}{\PYZsq{}}\PY{p}{)}
\end{Verbatim}
\end{tcolorbox}

    Aplicando um deslocamento inicial de 2 cm (além da flecha) na ponta da
régua, são medidas as acelerações:

    \begin{tcolorbox}[breakable, size=fbox, boxrule=1pt, pad at break*=1mm,colback=cellbackground, colframe=cellborder]
\prompt{In}{incolor}{135}{\boxspacing}
\begin{Verbatim}[commandchars=\\\{\}]
\PY{n}{data} \PY{o}{=} \PY{n}{Acelr}\PY{p}{(}\PY{l+m+mi}{15\PYZus{}000}\PY{p}{)}
\end{Verbatim}
\end{tcolorbox}

    \begin{Verbatim}[commandchars=\\\{\}]
Wait for it{\ldots}
Reading, go!
Acquisition ok!
Average sampling rate is 349.4Hz.
    \end{Verbatim}

    Plotando as acelerações (nas 3 direções) pelo tempo tem-se:

    \begin{tcolorbox}[breakable, size=fbox, boxrule=1pt, pad at break*=1mm,colback=cellbackground, colframe=cellborder]
\prompt{In}{incolor}{193}{\boxspacing}
\begin{Verbatim}[commandchars=\\\{\}]
\PY{n}{data}\PY{o}{.}\PY{n}{plot\PYZus{}time}\PY{p}{(}\PY{n}{figsize}\PY{o}{=}\PY{p}{(}\PY{l+m+mi}{12}\PY{p}{,}\PY{l+m+mi}{8}\PY{p}{)}\PY{p}{)}
\end{Verbatim}
\end{tcolorbox}

            \begin{tcolorbox}[breakable, size=fbox, boxrule=.5pt, pad at break*=1mm, opacityfill=0]
\prompt{Out}{outcolor}{193}{\boxspacing}
\begin{Verbatim}[commandchars=\\\{\}]
[[<matplotlib.lines.Line2D at 0x2b101b25400>],
 [<matplotlib.lines.Line2D at 0x2b101b51b80>],
 [<matplotlib.lines.Line2D at 0x2b101b8f340>]]
\end{Verbatim}
\end{tcolorbox}
        
    \begin{center}
    \adjustimage{max size={0.9\linewidth}{0.9\paperheight}}{../latex/trab5_files/../latex/trab5_8_1.pdf}
    \end{center}
    { \hspace*{\fill} \\}
    
    Nos segundos iniciais, previamente à excitação, observam-se acelerações
constantes e diferentes de zero nos três eixos. Essas se devem
principalmente à inclinação do equipamento na ponta da régua, que
apresenta uma flecha de cerca de 1 cm apenas com o peso da régua, do
acelerômetro e do cabo.

A fim de medir apenas acelerações relativas ao movimento dinâmico da
régua é subtraído de cada eixo seu valor médio, assim:

    \begin{tcolorbox}[breakable, size=fbox, boxrule=1pt, pad at break*=1mm,colback=cellbackground, colframe=cellborder]
\prompt{In}{incolor}{191}{\boxspacing}
\begin{Verbatim}[commandchars=\\\{\}]
\PY{n}{mean} \PY{o}{=} \PY{n}{data}\PY{o}{.}\PY{n}{mean}\PY{p}{(}\PY{n}{axis}\PY{o}{=}\PY{l+m+mi}{1}\PY{p}{)}
\PY{n}{acel} \PY{o}{=} \PY{n}{data} \PY{o}{\PYZhy{}} \PY{n}{np}\PY{o}{.}\PY{n}{array}\PY{p}{(}\PY{p}{[}\PY{n}{mean}\PY{p}{]}\PY{p}{)}\PY{o}{.}\PY{n}{T}
\PY{n}{acel}\PY{o}{.}\PY{n}{plot\PYZus{}time}\PY{p}{(}\PY{p}{)}
\end{Verbatim}
\end{tcolorbox}

            \begin{tcolorbox}[breakable, size=fbox, boxrule=.5pt, pad at break*=1mm, opacityfill=0]
\prompt{Out}{outcolor}{191}{\boxspacing}
\begin{Verbatim}[commandchars=\\\{\}]
[[<matplotlib.lines.Line2D at 0x2b100a560a0>],
 [<matplotlib.lines.Line2D at 0x2b100a82820>],
 [<matplotlib.lines.Line2D at 0x2b100ab2fa0>]]
\end{Verbatim}
\end{tcolorbox}
        
    \begin{center}
    \adjustimage{max size={0.9\linewidth}{0.9\paperheight}}{../latex/trab5_files/../latex/trab5_10_1.pdf}
    \end{center}
    { \hspace*{\fill} \\}
    
    Desprezando as acelerações no segundo eixo (Y), perpendicular ao plano
de interesse, o movimento oscilatório da ponta da régua é observado nos
eixos Z (vertical) e X (axial).

    \hypertarget{determinauxe7uxe3o-da-frequuxeancia-fundamental}{%
\section{Determinação da frequência
fundamental}\label{determinauxe7uxe3o-da-frequuxeancia-fundamental}}

Para determinação da frequência fundamental do sistema são plotadas as
densidades espectrais das acelerações medidas, obtidas pelo
periodograma:

    \begin{tcolorbox}[breakable, size=fbox, boxrule=1pt, pad at break*=1mm,colback=cellbackground, colframe=cellborder]
\prompt{In}{incolor}{194}{\boxspacing}
\begin{Verbatim}[commandchars=\\\{\}]
\PY{n}{acel}\PY{o}{.}\PY{n}{plot\PYZus{}freq}\PY{p}{(}\PY{p}{)}
\end{Verbatim}
\end{tcolorbox}

            \begin{tcolorbox}[breakable, size=fbox, boxrule=.5pt, pad at break*=1mm, opacityfill=0]
\prompt{Out}{outcolor}{194}{\boxspacing}
\begin{Verbatim}[commandchars=\\\{\}]
[[<matplotlib.lines.Line2D at 0x2b101c8a220>],
 [<matplotlib.lines.Line2D at 0x2b101c23070>],
 [<matplotlib.lines.Line2D at 0x2b101c7eac0>]]
\end{Verbatim}
\end{tcolorbox}
        
    \begin{center}
    \adjustimage{max size={0.9\linewidth}{0.9\paperheight}}{../latex/trab5_files/../latex/trab5_13_1.pdf}
    \end{center}
    { \hspace*{\fill} \\}
    
    Observa-se que apenas um modo de vibração, de frequência de cerca de
2Hz, foi excitado. Aproximando a região tem-se:

    \begin{tcolorbox}[breakable, size=fbox, boxrule=1pt, pad at break*=1mm,colback=cellbackground, colframe=cellborder]
\prompt{In}{incolor}{190}{\boxspacing}
\begin{Verbatim}[commandchars=\\\{\}]
\PY{n}{acel}\PY{o}{.}\PY{n}{plot\PYZus{}freq}\PY{p}{(}\PY{n}{axis\PYZus{}f}\PY{o}{=}\PY{p}{(}\PY{l+m+mi}{1}\PY{p}{,}\PY{l+m+mi}{4}\PY{p}{,}\PY{l+m+mi}{0}\PY{p}{,}\PY{l+m+mi}{25}\PY{p}{)}\PY{p}{)}
\end{Verbatim}
\end{tcolorbox}

            \begin{tcolorbox}[breakable, size=fbox, boxrule=.5pt, pad at break*=1mm, opacityfill=0]
\prompt{Out}{outcolor}{190}{\boxspacing}
\begin{Verbatim}[commandchars=\\\{\}]
[[<matplotlib.lines.Line2D at 0x2b100950280>],
 [<matplotlib.lines.Line2D at 0x2b10097ea30>],
 [<matplotlib.lines.Line2D at 0x2b1009bb220>]]
\end{Verbatim}
\end{tcolorbox}
        
    \begin{center}
    \adjustimage{max size={0.9\linewidth}{0.9\paperheight}}{../latex/trab5_files/../latex/trab5_15_1.pdf}
    \end{center}
    { \hspace*{\fill} \\}
    
    Logo a frequência fundamental é cerca de \textbf{2.6 Hz. }

    \hypertarget{determinauxe7uxe3o-do-deslocamento}{%
\section{Determinação do
deslocamento}\label{determinauxe7uxe3o-do-deslocamento}}

Integrando as acelerações medidas duas vezes obtem-se os deslocamentos
na ponta da régua. Filtrando apenas sinais no intervalo de 1 a 5 Hz
tem-se:

    \begin{tcolorbox}[breakable, size=fbox, boxrule=1pt, pad at break*=1mm,colback=cellbackground, colframe=cellborder]
\prompt{In}{incolor}{153}{\boxspacing}
\begin{Verbatim}[commandchars=\\\{\}]
\PY{n}{banda} \PY{o}{=} \PY{p}{[}\PY{l+m+mi}{1}\PY{p}{,} \PY{l+m+mi}{5}\PY{p}{]}
\PY{n}{desl} \PY{o}{=} \PY{p}{(}\PY{n}{acel}\PY{o}{.}\PY{n}{integrate}\PY{p}{(}\PY{n}{band}\PY{o}{=}\PY{n}{banda}\PY{p}{)}\PY{p}{)}\PY{o}{.}\PY{n}{integrate}\PY{p}{(}\PY{n}{band}\PY{o}{=}\PY{n}{banda}\PY{p}{)}
\PY{n}{desl}\PY{o}{.}\PY{n}{plot\PYZus{}time}\PY{p}{(}\PY{p}{)}
\end{Verbatim}
\end{tcolorbox}

            \begin{tcolorbox}[breakable, size=fbox, boxrule=.5pt, pad at break*=1mm, opacityfill=0]
\prompt{Out}{outcolor}{153}{\boxspacing}
\begin{Verbatim}[commandchars=\\\{\}]
[[<matplotlib.lines.Line2D at 0x2b17d077550>],
 [<matplotlib.lines.Line2D at 0x2b17d0a3cd0>],
 [<matplotlib.lines.Line2D at 0x2b17d0e0490>]]
\end{Verbatim}
\end{tcolorbox}
        
    \begin{center}
    \adjustimage{max size={0.9\linewidth}{0.9\paperheight}}{../latex/trab5_files/../latex/trab5_18_1.pdf}
    \end{center}
    { \hspace*{\fill} \\}
    
    Observam-se deslocamentos de cerca de 2 cm, conforme o deslocamento
inicial imposto.


    % Add a bibliography block to the postdoc
    
    
    
\end{document}
