\documentclass[a4paper,12pt]{article}

	%\usepackage[latin1]{inputenc}
	\usepackage[portuguese]{babel}

    \usepackage[breakable]{tcolorbox}
    \usepackage{parskip} % Stop auto-indenting (to mimic markdown behaviour)
    \usepackage{textgreek}

    
    
    \usepackage{iftex}
    \ifPDFTeX
    	\usepackage[T1]{fontenc}
    	\usepackage{mathpazo}
    \else
    	\usepackage{fontspec}
    \fi

    % Basic figure setup, for now with no caption control since it's done
    % automatically by Pandoc (which extracts ![](path) syntax from Markdown).
    \usepackage{graphicx}
    % Maintain compatibility with old templates. Remove in nbconvert 6.0
    \let\Oldincludegraphics\includegraphics
    % Ensure that by default, figures have no caption (until we provide a
    % proper Figure object with a Caption API and a way to capture that
    % in the conversion process - todo).
    \usepackage[justification=centering]{caption}
    %\DeclareCaptionFormat{nocaption}{}
    \captionsetup{aboveskip=0pt,belowskip=0pt}

    \usepackage{float}
    \floatplacement{figure}{H} % forces figures to be placed at the correct location
    \floatplacement{table}{H} % forces tables to be placed at the correct location
    \usepackage{xcolor} % Allow colors to be defined
    \usepackage{enumerate} % Needed for markdown enumerations to work
    \usepackage{geometry} % Used to adjust the document margins
    \usepackage{amsmath} % Equations
    \usepackage{amssymb} % Equations
    \usepackage{textcomp} % defines textquotesingle
    % Hack from http://tex.stackexchange.com/a/47451/13684:
    \AtBeginDocument{%
        \def\PYZsq{\textquotesingle}% Upright quotes in Pygmentized code
    }
    \usepackage{upquote} % Upright quotes for verbatim code
    \usepackage{eurosym} % defines \euro
    \usepackage[mathletters]{ucs} % Extended unicode (utf-8) support
    \usepackage{fancyvrb} % verbatim replacement that allows latex
    \usepackage{grffile} % extends the file name processing of package graphics 
                         % to support a larger range
    \makeatletter % fix for old versions of grffile with XeLaTeX
    \@ifpackagelater{grffile}{2019/11/01}
    {
      % Do nothing on new versions
    }
    {
      \def\Gread@@xetex#1{%
        \IfFileExists{"\Gin@base".bb}%
        {\Gread@eps{\Gin@base.bb}}%
        {\Gread@@xetex@aux#1}%
      }
    }
    \makeatother
    \usepackage[Export]{adjustbox} % Used to constrain images to a maximum size
    \adjustboxset{max size={0.9\linewidth}{0.9\paperheight}}

    % The hyperref package gives us a pdf with properly built
    % internal navigation ('pdf bookmarks' for the table of contents,
    % internal cross-reference links, web links for URLs, etc.)
    \usepackage{hyperref}
    % The default LaTeX title has an obnoxious amount of whitespace. By default,
    % titling removes some of it. It also provides customization options.
    \usepackage{titling}
    \usepackage{longtable} % longtable support required by pandoc >1.10
    \usepackage{booktabs}  % table support for pandoc > 1.12.2
    \usepackage[inline]{enumitem} % IRkernel/repr support (it uses the enumerate* environment)
    \usepackage[normalem]{ulem} % ulem is needed to support strikethroughs (\sout)
                                % normalem makes italics be italics, not underlines
    \usepackage{mathrsfs}
    

    
    % Colors for the hyperref package
    \definecolor{urlcolor}{rgb}{0,0,0} %{0,0,0.5} é azul escuro
    \definecolor{linkcolor}{rgb}{0,0,0}
    \definecolor{citecolor}{rgb}{0,0,0}

    % ANSI colors
    \definecolor{ansi-black}{HTML}{3E424D}
    \definecolor{ansi-black-intense}{HTML}{282C36}
    \definecolor{ansi-red}{HTML}{E75C58}
    \definecolor{ansi-red-intense}{HTML}{B22B31}
    \definecolor{ansi-green}{HTML}{00A250}
    \definecolor{ansi-green-intense}{HTML}{007427}
    \definecolor{ansi-yellow}{HTML}{DDB62B}
    \definecolor{ansi-yellow-intense}{HTML}{B27D12}
    \definecolor{ansi-blue}{HTML}{208FFB}
    \definecolor{ansi-blue-intense}{HTML}{0065CA}
    \definecolor{ansi-magenta}{HTML}{D160C4}
    \definecolor{ansi-magenta-intense}{HTML}{A03196}
    \definecolor{ansi-cyan}{HTML}{60C6C8}
    \definecolor{ansi-cyan-intense}{HTML}{258F8F}
    \definecolor{ansi-white}{HTML}{C5C1B4}
    \definecolor{ansi-white-intense}{HTML}{A1A6B2}
    \definecolor{ansi-default-inverse-fg}{HTML}{FFFFFF}
    \definecolor{ansi-default-inverse-bg}{HTML}{000000}

    % common color for the border for error outputs.
    \definecolor{outerrorbackground}{HTML}{FFDFDF}

    % commands and environments needed by pandoc snippets
    % extracted from the output of `pandoc -s`
    \providecommand{\tightlist}{%
      \setlength{\itemsep}{0pt}\setlength{\parskip}{0pt}}
    \DefineVerbatimEnvironment{Highlighting}{Verbatim}{commandchars=\\\{\}}
    % Add ',fontsize=\small' for more characters per line
    \newenvironment{Shaded}{}{}
    \newcommand{\KeywordTok}[1]{\textcolor[rgb]{0.00,0.44,0.13}{\textbf{{#1}}}}
    \newcommand{\DataTypeTok}[1]{\textcolor[rgb]{0.56,0.13,0.00}{{#1}}}
    \newcommand{\DecValTok}[1]{\textcolor[rgb]{0.25,0.63,0.44}{{#1}}}
    \newcommand{\BaseNTok}[1]{\textcolor[rgb]{0.25,0.63,0.44}{{#1}}}
    \newcommand{\FloatTok}[1]{\textcolor[rgb]{0.25,0.63,0.44}{{#1}}}
    \newcommand{\CharTok}[1]{\textcolor[rgb]{0.25,0.44,0.63}{{#1}}}
    \newcommand{\StringTok}[1]{\textcolor[rgb]{0.25,0.44,0.63}{{#1}}}
    \newcommand{\CommentTok}[1]{\textcolor[rgb]{0.38,0.63,0.69}{\textit{{#1}}}}
    \newcommand{\OtherTok}[1]{\textcolor[rgb]{0.00,0.44,0.13}{{#1}}}
    \newcommand{\AlertTok}[1]{\textcolor[rgb]{1.00,0.00,0.00}{\textbf{{#1}}}}
    \newcommand{\FunctionTok}[1]{\textcolor[rgb]{0.02,0.16,0.49}{{#1}}}
    \newcommand{\RegionMarkerTok}[1]{{#1}}
    \newcommand{\ErrorTok}[1]{\textcolor[rgb]{1.00,0.00,0.00}{\textbf{{#1}}}}
    \newcommand{\NormalTok}[1]{{#1}}
    
    % Additional commands for more recent versions of Pandoc
    \newcommand{\ConstantTok}[1]{\textcolor[rgb]{0.53,0.00,0.00}{{#1}}}
    \newcommand{\SpecialCharTok}[1]{\textcolor[rgb]{0.25,0.44,0.63}{{#1}}}
    \newcommand{\VerbatimStringTok}[1]{\textcolor[rgb]{0.25,0.44,0.63}{{#1}}}
    \newcommand{\SpecialStringTok}[1]{\textcolor[rgb]{0.73,0.40,0.53}{{#1}}}
    \newcommand{\ImportTok}[1]{{#1}}
    \newcommand{\DocumentationTok}[1]{\textcolor[rgb]{0.73,0.13,0.13}{\textit{{#1}}}}
    \newcommand{\AnnotationTok}[1]{\textcolor[rgb]{0.38,0.63,0.69}{\textbf{\textit{{#1}}}}}
    \newcommand{\CommentVarTok}[1]{\textcolor[rgb]{0.38,0.63,0.69}{\textbf{\textit{{#1}}}}}
    \newcommand{\VariableTok}[1]{\textcolor[rgb]{0.10,0.09,0.49}{{#1}}}
    \newcommand{\ControlFlowTok}[1]{\textcolor[rgb]{0.00,0.44,0.13}{\textbf{{#1}}}}
    \newcommand{\OperatorTok}[1]{\textcolor[rgb]{0.40,0.40,0.40}{{#1}}}
    \newcommand{\BuiltInTok}[1]{{#1}}
    \newcommand{\ExtensionTok}[1]{{#1}}
    \newcommand{\PreprocessorTok}[1]{\textcolor[rgb]{0.74,0.48,0.00}{{#1}}}
    \newcommand{\AttributeTok}[1]{\textcolor[rgb]{0.49,0.56,0.16}{{#1}}}
    \newcommand{\InformationTok}[1]{\textcolor[rgb]{0.38,0.63,0.69}{\textbf{\textit{{#1}}}}}
    \newcommand{\WarningTok}[1]{\textcolor[rgb]{0.38,0.63,0.69}{\textbf{\textit{{#1}}}}}
    
    
    % Define a nice break command that doesn't care if a line doesn't already
    % exist.
    \def\br{\hspace*{\fill} \\* }
    % Math Jax compatibility definitions
    \def\gt{>}
    \def\lt{<}
    \let\Oldtex\TeX
    \let\Oldlatex\LaTeX
    \renewcommand{\TeX}{\textrm{\Oldtex}}
    \renewcommand{\LaTeX}{\textrm{\Oldlatex}}
   
    
    
    
    
    
% Pygments definitions
\makeatletter
\def\PY@reset{\let\PY@it=\relax \let\PY@bf=\relax%
    \let\PY@ul=\relax \let\PY@tc=\relax%
    \let\PY@bc=\relax \let\PY@ff=\relax}
\def\PY@tok#1{\csname PY@tok@#1\endcsname}
\def\PY@toks#1+{\ifx\relax#1\empty\else%
    \PY@tok{#1}\expandafter\PY@toks\fi}
\def\PY@do#1{\PY@bc{\PY@tc{\PY@ul{%
    \PY@it{\PY@bf{\PY@ff{#1}}}}}}}
\def\PY#1#2{\PY@reset\PY@toks#1+\relax+\PY@do{#2}}

\expandafter\def\csname PY@tok@w\endcsname{\def\PY@tc##1{\textcolor[rgb]{0.73,0.73,0.73}{##1}}}
\expandafter\def\csname PY@tok@c\endcsname{\let\PY@it=\textit\def\PY@tc##1{\textcolor[rgb]{0.25,0.50,0.50}{##1}}}
\expandafter\def\csname PY@tok@cp\endcsname{\def\PY@tc##1{\textcolor[rgb]{0.74,0.48,0.00}{##1}}}
\expandafter\def\csname PY@tok@k\endcsname{\let\PY@bf=\textbf\def\PY@tc##1{\textcolor[rgb]{0.00,0.50,0.00}{##1}}}
\expandafter\def\csname PY@tok@kp\endcsname{\def\PY@tc##1{\textcolor[rgb]{0.00,0.50,0.00}{##1}}}
\expandafter\def\csname PY@tok@kt\endcsname{\def\PY@tc##1{\textcolor[rgb]{0.69,0.00,0.25}{##1}}}
\expandafter\def\csname PY@tok@o\endcsname{\def\PY@tc##1{\textcolor[rgb]{0.40,0.40,0.40}{##1}}}
\expandafter\def\csname PY@tok@ow\endcsname{\let\PY@bf=\textbf\def\PY@tc##1{\textcolor[rgb]{0.67,0.13,1.00}{##1}}}
\expandafter\def\csname PY@tok@nb\endcsname{\def\PY@tc##1{\textcolor[rgb]{0.00,0.50,0.00}{##1}}}
\expandafter\def\csname PY@tok@nf\endcsname{\def\PY@tc##1{\textcolor[rgb]{0.00,0.00,1.00}{##1}}}
\expandafter\def\csname PY@tok@nc\endcsname{\let\PY@bf=\textbf\def\PY@tc##1{\textcolor[rgb]{0.00,0.00,1.00}{##1}}}
\expandafter\def\csname PY@tok@nn\endcsname{\let\PY@bf=\textbf\def\PY@tc##1{\textcolor[rgb]{0.00,0.00,1.00}{##1}}}
\expandafter\def\csname PY@tok@ne\endcsname{\let\PY@bf=\textbf\def\PY@tc##1{\textcolor[rgb]{0.82,0.25,0.23}{##1}}}
\expandafter\def\csname PY@tok@nv\endcsname{\def\PY@tc##1{\textcolor[rgb]{0.10,0.09,0.49}{##1}}}
\expandafter\def\csname PY@tok@no\endcsname{\def\PY@tc##1{\textcolor[rgb]{0.53,0.00,0.00}{##1}}}
\expandafter\def\csname PY@tok@nl\endcsname{\def\PY@tc##1{\textcolor[rgb]{0.63,0.63,0.00}{##1}}}
\expandafter\def\csname PY@tok@ni\endcsname{\let\PY@bf=\textbf\def\PY@tc##1{\textcolor[rgb]{0.60,0.60,0.60}{##1}}}
\expandafter\def\csname PY@tok@na\endcsname{\def\PY@tc##1{\textcolor[rgb]{0.49,0.56,0.16}{##1}}}
\expandafter\def\csname PY@tok@nt\endcsname{\let\PY@bf=\textbf\def\PY@tc##1{\textcolor[rgb]{0.00,0.50,0.00}{##1}}}
\expandafter\def\csname PY@tok@nd\endcsname{\def\PY@tc##1{\textcolor[rgb]{0.67,0.13,1.00}{##1}}}
\expandafter\def\csname PY@tok@s\endcsname{\def\PY@tc##1{\textcolor[rgb]{0.73,0.13,0.13}{##1}}}
\expandafter\def\csname PY@tok@sd\endcsname{\let\PY@it=\textit\def\PY@tc##1{\textcolor[rgb]{0.73,0.13,0.13}{##1}}}
\expandafter\def\csname PY@tok@si\endcsname{\let\PY@bf=\textbf\def\PY@tc##1{\textcolor[rgb]{0.73,0.40,0.53}{##1}}}
\expandafter\def\csname PY@tok@se\endcsname{\let\PY@bf=\textbf\def\PY@tc##1{\textcolor[rgb]{0.73,0.40,0.13}{##1}}}
\expandafter\def\csname PY@tok@sr\endcsname{\def\PY@tc##1{\textcolor[rgb]{0.73,0.40,0.53}{##1}}}
\expandafter\def\csname PY@tok@ss\endcsname{\def\PY@tc##1{\textcolor[rgb]{0.10,0.09,0.49}{##1}}}
\expandafter\def\csname PY@tok@sx\endcsname{\def\PY@tc##1{\textcolor[rgb]{0.00,0.50,0.00}{##1}}}
\expandafter\def\csname PY@tok@m\endcsname{\def\PY@tc##1{\textcolor[rgb]{0.40,0.40,0.40}{##1}}}
\expandafter\def\csname PY@tok@gh\endcsname{\let\PY@bf=\textbf\def\PY@tc##1{\textcolor[rgb]{0.00,0.00,0.50}{##1}}}
\expandafter\def\csname PY@tok@gu\endcsname{\let\PY@bf=\textbf\def\PY@tc##1{\textcolor[rgb]{0.50,0.00,0.50}{##1}}}
\expandafter\def\csname PY@tok@gd\endcsname{\def\PY@tc##1{\textcolor[rgb]{0.63,0.00,0.00}{##1}}}
\expandafter\def\csname PY@tok@gi\endcsname{\def\PY@tc##1{\textcolor[rgb]{0.00,0.63,0.00}{##1}}}
\expandafter\def\csname PY@tok@gr\endcsname{\def\PY@tc##1{\textcolor[rgb]{1.00,0.00,0.00}{##1}}}
\expandafter\def\csname PY@tok@ge\endcsname{\let\PY@it=\textit}
\expandafter\def\csname PY@tok@gs\endcsname{\let\PY@bf=\textbf}
\expandafter\def\csname PY@tok@gp\endcsname{\let\PY@bf=\textbf\def\PY@tc##1{\textcolor[rgb]{0.00,0.00,0.50}{##1}}}
\expandafter\def\csname PY@tok@go\endcsname{\def\PY@tc##1{\textcolor[rgb]{0.53,0.53,0.53}{##1}}}
\expandafter\def\csname PY@tok@gt\endcsname{\def\PY@tc##1{\textcolor[rgb]{0.00,0.27,0.87}{##1}}}
\expandafter\def\csname PY@tok@err\endcsname{\def\PY@bc##1{\setlength{\fboxsep}{0pt}\fcolorbox[rgb]{1.00,0.00,0.00}{1,1,1}{\strut ##1}}}
\expandafter\def\csname PY@tok@kc\endcsname{\let\PY@bf=\textbf\def\PY@tc##1{\textcolor[rgb]{0.00,0.50,0.00}{##1}}}
\expandafter\def\csname PY@tok@kd\endcsname{\let\PY@bf=\textbf\def\PY@tc##1{\textcolor[rgb]{0.00,0.50,0.00}{##1}}}
\expandafter\def\csname PY@tok@kn\endcsname{\let\PY@bf=\textbf\def\PY@tc##1{\textcolor[rgb]{0.00,0.50,0.00}{##1}}}
\expandafter\def\csname PY@tok@kr\endcsname{\let\PY@bf=\textbf\def\PY@tc##1{\textcolor[rgb]{0.00,0.50,0.00}{##1}}}
\expandafter\def\csname PY@tok@bp\endcsname{\def\PY@tc##1{\textcolor[rgb]{0.00,0.50,0.00}{##1}}}
\expandafter\def\csname PY@tok@fm\endcsname{\def\PY@tc##1{\textcolor[rgb]{0.00,0.00,1.00}{##1}}}
\expandafter\def\csname PY@tok@vc\endcsname{\def\PY@tc##1{\textcolor[rgb]{0.10,0.09,0.49}{##1}}}
\expandafter\def\csname PY@tok@vg\endcsname{\def\PY@tc##1{\textcolor[rgb]{0.10,0.09,0.49}{##1}}}
\expandafter\def\csname PY@tok@vi\endcsname{\def\PY@tc##1{\textcolor[rgb]{0.10,0.09,0.49}{##1}}}
\expandafter\def\csname PY@tok@vm\endcsname{\def\PY@tc##1{\textcolor[rgb]{0.10,0.09,0.49}{##1}}}
\expandafter\def\csname PY@tok@sa\endcsname{\def\PY@tc##1{\textcolor[rgb]{0.73,0.13,0.13}{##1}}}
\expandafter\def\csname PY@tok@sb\endcsname{\def\PY@tc##1{\textcolor[rgb]{0.73,0.13,0.13}{##1}}}
\expandafter\def\csname PY@tok@sc\endcsname{\def\PY@tc##1{\textcolor[rgb]{0.73,0.13,0.13}{##1}}}
\expandafter\def\csname PY@tok@dl\endcsname{\def\PY@tc##1{\textcolor[rgb]{0.73,0.13,0.13}{##1}}}
\expandafter\def\csname PY@tok@s2\endcsname{\def\PY@tc##1{\textcolor[rgb]{0.73,0.13,0.13}{##1}}}
\expandafter\def\csname PY@tok@sh\endcsname{\def\PY@tc##1{\textcolor[rgb]{0.73,0.13,0.13}{##1}}}
\expandafter\def\csname PY@tok@s1\endcsname{\def\PY@tc##1{\textcolor[rgb]{0.73,0.13,0.13}{##1}}}
\expandafter\def\csname PY@tok@mb\endcsname{\def\PY@tc##1{\textcolor[rgb]{0.40,0.40,0.40}{##1}}}
\expandafter\def\csname PY@tok@mf\endcsname{\def\PY@tc##1{\textcolor[rgb]{0.40,0.40,0.40}{##1}}}
\expandafter\def\csname PY@tok@mh\endcsname{\def\PY@tc##1{\textcolor[rgb]{0.40,0.40,0.40}{##1}}}
\expandafter\def\csname PY@tok@mi\endcsname{\def\PY@tc##1{\textcolor[rgb]{0.40,0.40,0.40}{##1}}}
\expandafter\def\csname PY@tok@il\endcsname{\def\PY@tc##1{\textcolor[rgb]{0.40,0.40,0.40}{##1}}}
\expandafter\def\csname PY@tok@mo\endcsname{\def\PY@tc##1{\textcolor[rgb]{0.40,0.40,0.40}{##1}}}
\expandafter\def\csname PY@tok@ch\endcsname{\let\PY@it=\textit\def\PY@tc##1{\textcolor[rgb]{0.25,0.50,0.50}{##1}}}
\expandafter\def\csname PY@tok@cm\endcsname{\let\PY@it=\textit\def\PY@tc##1{\textcolor[rgb]{0.25,0.50,0.50}{##1}}}
\expandafter\def\csname PY@tok@cpf\endcsname{\let\PY@it=\textit\def\PY@tc##1{\textcolor[rgb]{0.25,0.50,0.50}{##1}}}
\expandafter\def\csname PY@tok@c1\endcsname{\let\PY@it=\textit\def\PY@tc##1{\textcolor[rgb]{0.25,0.50,0.50}{##1}}}
\expandafter\def\csname PY@tok@cs\endcsname{\let\PY@it=\textit\def\PY@tc##1{\textcolor[rgb]{0.25,0.50,0.50}{##1}}}

\def\PYZbs{\char`\\}
\def\PYZus{\char`\_}
\def\PYZob{\char`\{}
\def\PYZcb{\char`\}}
\def\PYZca{\char`\^}
\def\PYZam{\char`\&}
\def\PYZlt{\char`\<}
\def\PYZgt{\char`\>}
\def\PYZsh{\char`\#}
\def\PYZpc{\char`\%}
\def\PYZdl{\char`\$}
\def\PYZhy{\char`\-}
\def\PYZsq{\char`\'}
\def\PYZdq{\char`\"}
\def\PYZti{\char`\~}
% for compatibility with earlier versions
\def\PYZat{@}
\def\PYZlb{[}
\def\PYZrb{]}
\makeatother


    % For linebreaks inside Verbatim environment from package fancyvrb. 
    \makeatletter
        \newbox\Wrappedcontinuationbox 
        \newbox\Wrappedvisiblespacebox 
        \newcommand*\Wrappedvisiblespace {\textcolor{red}{\textvisiblespace}} 
        \newcommand*\Wrappedcontinuationsymbol {\textcolor{red}{\llap{\tiny$\m@th\hookrightarrow$}}} 
        \newcommand*\Wrappedcontinuationindent {3ex } 
        \newcommand*\Wrappedafterbreak {\kern\Wrappedcontinuationindent\copy\Wrappedcontinuationbox} 
        % Take advantage of the already applied Pygments mark-up to insert 
        % potential linebreaks for TeX processing. 
        %        {, <, #, %, $, ' and ": go to next line. 
        %        _, }, ^, &, >, - and ~: stay at end of broken line. 
        % Use of \textquotesingle for straight quote. 
        \newcommand*\Wrappedbreaksatspecials {% 
            \def\PYGZus{\discretionary{\char`\_}{\Wrappedafterbreak}{\char`\_}}% 
            \def\PYGZob{\discretionary{}{\Wrappedafterbreak\char`\{}{\char`\{}}% 
            \def\PYGZcb{\discretionary{\char`\}}{\Wrappedafterbreak}{\char`\}}}% 
            \def\PYGZca{\discretionary{\char`\^}{\Wrappedafterbreak}{\char`\^}}% 
            \def\PYGZam{\discretionary{\char`\&}{\Wrappedafterbreak}{\char`\&}}% 
            \def\PYGZlt{\discretionary{}{\Wrappedafterbreak\char`\<}{\char`\<}}% 
            \def\PYGZgt{\discretionary{\char`\>}{\Wrappedafterbreak}{\char`\>}}% 
            \def\PYGZsh{\discretionary{}{\Wrappedafterbreak\char`\#}{\char`\#}}% 
            \def\PYGZpc{\discretionary{}{\Wrappedafterbreak\char`\%}{\char`\%}}% 
            \def\PYGZdl{\discretionary{}{\Wrappedafterbreak\char`\$}{\char`\$}}% 
            \def\PYGZhy{\discretionary{\char`\-}{\Wrappedafterbreak}{\char`\-}}% 
            \def\PYGZsq{\discretionary{}{\Wrappedafterbreak\textquotesingle}{\textquotesingle}}% 
            \def\PYGZdq{\discretionary{}{\Wrappedafterbreak\char`\"}{\char`\"}}% 
            \def\PYGZti{\discretionary{\char`\~}{\Wrappedafterbreak}{\char`\~}}% 
        } 
        % Some characters . , ; ? ! / are not pygmentized. 
        % This macro makes them "active" and they will insert potential linebreaks 
        \newcommand*\Wrappedbreaksatpunct {% 
            \lccode`\~`\.\lowercase{\def~}{\discretionary{\hbox{\char`\.}}{\Wrappedafterbreak}{\hbox{\char`\.}}}% 
            \lccode`\~`\,\lowercase{\def~}{\discretionary{\hbox{\char`\,}}{\Wrappedafterbreak}{\hbox{\char`\,}}}% 
            \lccode`\~`\;\lowercase{\def~}{\discretionary{\hbox{\char`\;}}{\Wrappedafterbreak}{\hbox{\char`\;}}}% 
            \lccode`\~`\:\lowercase{\def~}{\discretionary{\hbox{\char`\:}}{\Wrappedafterbreak}{\hbox{\char`\:}}}% 
            \lccode`\~`\?\lowercase{\def~}{\discretionary{\hbox{\char`\?}}{\Wrappedafterbreak}{\hbox{\char`\?}}}% 
            \lccode`\~`\!\lowercase{\def~}{\discretionary{\hbox{\char`\!}}{\Wrappedafterbreak}{\hbox{\char`\!}}}% 
            \lccode`\~`\/\lowercase{\def~}{\discretionary{\hbox{\char`\/}}{\Wrappedafterbreak}{\hbox{\char`\/}}}% 
            \catcode`\.\active
            \catcode`\,\active 
            \catcode`\;\active
            \catcode`\:\active
            \catcode`\?\active
            \catcode`\!\active
            \catcode`\/\active 
            \lccode`\~`\~ 	
        }
    \makeatother

    \let\OriginalVerbatim=\Verbatim
    \makeatletter
    \renewcommand{\Verbatim}[1][1]{%
    	
    	%\fontfamily{fi4}
  		%\fontsize{11}{2}
  		\linespread{0.9}	
    	
        %\parskip\z@skip
        \sbox\Wrappedcontinuationbox {\Wrappedcontinuationsymbol}%
        \sbox\Wrappedvisiblespacebox {\FV@SetupFont\Wrappedvisiblespace}%
        \def\FancyVerbFormatLine ##1{\hsize\linewidth
            \vtop{\raggedright\hyphenpenalty\z@\exhyphenpenalty\z@
                \doublehyphendemerits\z@\finalhyphendemerits\z@
                \strut ##1\strut}%
        }%
        % If the linebreak is at a space, the latter will be displayed as visible
        % space at end of first line, and a continuation symbol starts next line.
        % Stretch/shrink are however usually zero for typewriter font.
        \def\FV@Space {%
            \nobreak\hskip\z@ plus\fontdimen3\font minus\fontdimen4\font
            \discretionary{\copy\Wrappedvisiblespacebox}{\Wrappedafterbreak}
            {\kern\fontdimen2\font}%
        }%
        
        % Allow breaks at special characters using \PYG... macros.
        \Wrappedbreaksatspecials
        % Breaks at punctuation characters . , ; ? ! and / need catcode=\active 	
        \OriginalVerbatim[#1,codes*=\Wrappedbreaksatpunct]%
    }
    \makeatother

    % Exact colors from NB
    \definecolor{incolor}{HTML}{303F9F}
    \definecolor{outcolor}{HTML}{D84315}
    \definecolor{cellborder}{HTML}{CFCFCF}
    \definecolor{cellbackground}{HTML}{F7F7F7}
    
    % prompt
    \makeatletter
    \newcommand{\boxspacing}{\kern\kvtcb@left@rule\kern\kvtcb@boxsep}
    \makeatother
    \newcommand{\prompt}[4]{ }
    

    
    % Prevent overflowing lines due to hard-to-break entities
    \sloppy 
    % Setup hyperref package
    \hypersetup{
      breaklinks=true,  % so long urls are correctly broken across lines
      colorlinks=true,
      urlcolor=urlcolor,
      linkcolor=linkcolor,
      citecolor=citecolor,
      }
    % Slightly bigger margins than the latex defaults
    
    \geometry{verbose,tmargin=1in,bmargin=1in,lmargin=1in,rmargin=1in}

 \linespread{1.25}
    
 
 \makeatletter
 
 \newcommand{\subtitle}[1]{\def\@subtitle{#1}}
 
 \renewcommand\maketitle{%
 		\begin{center}
 			{\large Universidade Federal do Rio Grande do Sul}\\
 			{\normalsize Escola de Engenharia}\\
 			{\normalsize Programa de Pós-Graduação em Engenharia Civil}\\[48pt] 
 			{\Large \@title }\\[8pt] 
 			{\large \@subtitle }\\[24pt] 
 		\end{center}
 		\begin{flushright}
 			{\normalsize \@author}\\[24pt] 
 		\end{flushright}
 		\begin{center}
 			{\@date}\\[0pt] 
 		\end{center}
 		\noindent\rule{17cm}{0.8pt}
 }
\makeatother

\author{Eduardo Pagnussat Titello}
\title{PEC00144 - Métodos Experimentais em Engenharia Civil}
\subtitle{Trabalho 1}
\date{Novembro de 2020}


\begin{document}
    
\maketitle
    
    
%%%\hypertarget{apresentauxe7uxe3o}{%
%%%\section{Apresentação}\label{apresentauxe7uxe3o}}

Este trabalho tem dois objetivos, logo está divido em duas partes:

\hyperref[parte-i]{Parte I}: pesquisar por um número \(\Pi\) importante,
escrever a matriz dimensional das grandezas envolvidas, confirmar a
adimensionalidade usando a fórmula \(\vec{\alpha}\,{\bf D}\) e descrever
exemplos de aplicação.

\hyperref[parte-ii]{Parte II}: definir um problema de interesse, para o
qual existe uma expressão \textbf{dimensional} conhecida e utilizada na
prática; reformular a expressão em termos de números \(\Pi\) propostos;
apresentar resultados na forma de gráficos relacionando estes números.

    \hypertarget{parte-i}{%
\section{Parte I}\label{parte-i}}

O número de Strouhal (\(St\)) é um importante adimensional da mecânica
dos fluídos que descreve oscilações no escoamento. Conforme White
(2017), mesmo escoamentos que aparentam estar perfeitamente em regime
permanente apresentam padrões de oscilação que variam conforme o número
de Reynolds (\(Re\), considerado por muitos o mais importante
adimensional da mecânica dos fluídos).

\(St\) relaciona a frequência de desprendimento de vortices (\(f\)), a
dimensão característica (\(L\)) e a velocidade do escoamento (\(v\))
através de:

\begin{equation}
St=\frac{fL}{v}
\end{equation}

onde para esferas e cilindros \(L\) é seu diâmetro.

    Fazendo uso da metologia adotada por Rocha (2020), a matriz dimensional
\(\bf D\) do problema é:

    \begin{tcolorbox}[breakable, size=fbox, boxrule=1pt, pad at break*=1mm,colback=cellbackground, colframe=cellborder]
\prompt{In}{incolor}{1}{\boxspacing}
\begin{Verbatim}[commandchars=\\\{\}]
\PY{c+c1}{\PYZsh{} Importando módulos}
\PY{k+kn}{import} \PY{n+nn}{numpy} \PY{k}{as} \PY{n+nn}{np}
\PY{k+kn}{import} \PY{n+nn}{pandas} \PY{k}{as} \PY{n+nn}{pd} 
\PY{k+kn}{import} \PY{n+nn}{matplotlib}\PY{n+nn}{.}\PY{n+nn}{pyplot} \PY{k}{as} \PY{n+nn}{plt}
\PY{k+kn}{import} \PY{n+nn}{matplotlib}
\PY{o}{\PYZpc{}}\PY{k}{config} InlineBackend.figure\PYZus{}format = \PYZsq{}svg\PYZsq{} \PYZsh{} Muda backend do jupyter para SVG ;)
\PY{k+kn}{import} \PY{n+nn}{jupyter2latex} \PY{k}{as} \PY{n+nn}{j2l} \PY{c+c1}{\PYZsh{} Uma maneira que encontrei para tabelas ficarem ok (github.com/dutitello/Jupyter2Latex)}

\PY{c+c1}{\PYZsh{} Importando todas grandezas}
\PY{n}{DimData} \PY{o}{=} \PY{n}{pd}\PY{o}{.}\PY{n}{read\PYZus{}excel}\PY{p}{(}\PY{l+s+s1}{\PYZsq{}}\PY{l+s+s1}{../resources/DimData.xlsx}\PY{l+s+s1}{\PYZsq{}}\PY{p}{,} 
                         \PY{n}{index\PYZus{}col}  \PY{o}{=}  \PY{l+m+mi}{0}\PY{p}{,}
                         \PY{n}{sheet\PYZus{}name} \PY{o}{=} \PY{l+s+s1}{\PYZsq{}}\PY{l+s+s1}{DimData}\PY{l+s+s1}{\PYZsq{}}\PY{p}{)}

\PY{c+c1}{\PYZsh{} Filtrando apenas as grandezas envolvídas no problema:}
\PY{n}{DMat} \PY{o}{=} \PY{n}{DimData}\PY{o}{.}\PY{n}{loc}\PY{p}{[}\PY{p}{[}\PY{l+s+s1}{\PYZsq{}}\PY{l+s+s1}{f}\PY{l+s+s1}{\PYZsq{}}\PY{p}{,} \PY{l+s+s1}{\PYZsq{}}\PY{l+s+s1}{L}\PY{l+s+s1}{\PYZsq{}}\PY{p}{,} \PY{l+s+s1}{\PYZsq{}}\PY{l+s+s1}{v}\PY{l+s+s1}{\PYZsq{}}\PY{p}{]}\PY{p}{,} \PY{p}{[}\PY{l+s+s1}{\PYZsq{}}\PY{l+s+s1}{L}\PY{l+s+s1}{\PYZsq{}}\PY{p}{,}\PY{l+s+s1}{\PYZsq{}}\PY{l+s+s1}{M}\PY{l+s+s1}{\PYZsq{}}\PY{p}{,}\PY{l+s+s1}{\PYZsq{}}\PY{l+s+s1}{T}\PY{l+s+s1}{\PYZsq{}}\PY{p}{]}\PY{p}{]}
\PY{n}{j2l}\PY{o}{.}\PY{n}{df2table}\PY{p}{(}\PY{n}{DMat}\PY{p}{,} \PY{n}{caption}\PY{o}{=}\PY{l+s+s1}{\PYZsq{}}\PY{l+s+s1}{Matriz D}\PY{l+s+s1}{\PYZsq{}}\PY{p}{)}
\end{Verbatim}
\end{tcolorbox}

    
    \begin{table}[h!]
    \centering
    \caption{Matriz D}
    {\begin{tabular}{lrrr}
\toprule
{} &  L &  M &  T \\
\midrule
f &  0 &  0 & -1 \\
L &  1 &  0 &  0 \\
v &  1 &  0 & -1 \\
\bottomrule
\end{tabular}
}
    \label{}
    \end{table}
    

    
    Que adotando o sistema internacional de unidades leva à: \begin{align*}
[ f ] &= {\rm m}^{0}{\rm kg}^{0}{\rm s}^{-1} = 1/{\rm s} = {\rm Hz} \\
[ L ] &= {\rm m}^{1}{\rm kg}^{0}{\rm s}^{0} = {\rm m}  \\
[ v ] &= {\rm m}^{1}{\rm kg}^{0}{\rm s}^{-1} = {\rm m/s} \\
\end{align*}

Logo a matriz \(\bf D\) está correta.

Reescrevendo \(St\) tem-se:

\begin{equation}
\Pi_{St}=f^1L^1v^{-1}
\end{equation}

logo: \begin{equation}
\vec\alpha_{St}=[1, 1, -1]
\end{equation}

Com isso o produto \(\vec\alpha_{St}.\bf D\) é:

    \begin{tcolorbox}[breakable, size=fbox, boxrule=1pt, pad at break*=1mm,colback=cellbackground, colframe=cellborder]
\prompt{In}{incolor}{2}{\boxspacing}
\begin{Verbatim}[commandchars=\\\{\}]
\PY{n}{alpha\PYZus{}St} \PY{o}{=} \PY{n}{np}\PY{o}{.}\PY{n}{array}\PY{p}{(}\PY{p}{[}\PY{l+m+mi}{1}\PY{p}{,} \PY{l+m+mi}{1}\PY{p}{,} \PY{o}{\PYZhy{}}\PY{l+m+mi}{1}\PY{p}{]}\PY{p}{)}
\PY{n}{prod} \PY{o}{=} \PY{n}{alpha\PYZus{}St}\PY{o}{.}\PY{n}{dot}\PY{p}{(}\PY{n}{DMat}\PY{o}{.}\PY{n}{values}\PY{p}{)}
\PY{n+nb}{print}\PY{p}{(}\PY{n}{prod}\PY{p}{)}
\end{Verbatim}
\end{tcolorbox}

    \begin{Verbatim}[commandchars=\\\{\}]
[0 0 0]
    \end{Verbatim}

    Logo a expressão de \(St\) apresentada é adimensional.

    Fenômenos relacionados ao número de Strouhal são o cantar de cabos com o
vento, o galope de linhas de ancoragem submarinas e as vibrações de
construções esbeltas com o vento; sendo que o nome Strouhal faz menção
ao físico V. Strouhal, que estudou o cantar a fios excitados pelo vento
em 1878. Esses fenomenos ocorrem pelo desprendimento periódico de
vórtices, denominados vórtices de van Kármán, em menção ao físico T. von
Kármán que os explicou teóricamente em 1912 (White, 2017).

Os vórtices de van Kármán podem levar estruturas à ressonancia caso sua
frequência de desprendimento se aproxime da frequencia de vibração livre
da estrutura, como ocorrido na ponte Tacoma em 1940. Em esferas, por
exemplo, os vórtices são desprendidos quando \(St \approx 0,21\) no
intervalo \(10^2<Re<10^7\) (White, 2017).

No Brasil, a norma NBR 6123:1988 - Forças devidas ao vento em
edificações, no anexo H, fornece valores para \(St\) com base em \(Re\)
e no tipo da seção da edificação, de forma a permitir a obtenção de uma
velocidade crítica na qual a frequência de desprendimento de um par de
vórtices coincide com uma das frequências naturais de vibração da
estrutura. Através desse anexo os projetistas podem então, mesmo que de
forma aproximada, avaliar o fenomeno de desprendimento de vórtices em
edificações.

    \hypertarget{parte-ii}{%
\section{Parte II}\label{parte-ii}}

    A expressão adotada, apresentada em \ref{expviga}, determina o momento
resistente de uma viga em concreto armado, com armadura simples de
tração, no domínio II ou III (armadura escoando, conforme norma
brasileira). Essa é obtida através do equilibrio de esforços internos da
seção considerando o diagrama retangular de tensões para o concreto.

\begin{equation}
M = A_s f_y \Bigg( d - \frac{A_s f_y}{2 b f_c}\Bigg)
\label{expviga}
\end{equation}

onde seus paramêtros são:

\begin{itemize}
\tightlist
\item
  \(M\): momento resistente da seção;
\item
  \(A_s\): área de armadura tracionada;
\item
  \(f_y\): tensão de escoamento da armadura;
\item
  \(d\): altura útil (do topo da viga ao centro da armadura);
\item
  \(b\): largura da viga;
\item
  \(f_c\): resistência à compressão do concreto.
\end{itemize}

    A matriz dimensional \(\bf{D}\) é então:

    \begin{tcolorbox}[breakable, size=fbox, boxrule=1pt, pad at break*=1mm,colback=cellbackground, colframe=cellborder]
\prompt{In}{incolor}{3}{\boxspacing}
\begin{Verbatim}[commandchars=\\\{\}]
\PY{n}{DMatCA} \PY{o}{=} \PY{n}{DimData}\PY{o}{.}\PY{n}{loc}\PY{p}{[}\PY{p}{[}\PY{l+s+s1}{\PYZsq{}}\PY{l+s+s1}{M}\PY{l+s+s1}{\PYZsq{}}\PY{p}{,} \PY{l+s+s1}{\PYZsq{}}\PY{l+s+s1}{A}\PY{l+s+s1}{\PYZsq{}}\PY{p}{,} \PY{l+s+s1}{\PYZsq{}}\PY{l+s+s1}{\textsigma}\PY{l+s+s1}{\PYZsq{}}\PY{p}{,} \PY{l+s+s1}{\PYZsq{}}\PY{l+s+s1}{L}\PY{l+s+s1}{\PYZsq{}}\PY{p}{,} \PY{l+s+s1}{\PYZsq{}}\PY{l+s+s1}{L}\PY{l+s+s1}{\PYZsq{}}\PY{p}{,} \PY{l+s+s1}{\PYZsq{}}\PY{l+s+s1}{\textsigma}\PY{l+s+s1}{\PYZsq{}}\PY{p}{]}\PY{p}{,} \PY{p}{[}\PY{l+s+s1}{\PYZsq{}}\PY{l+s+s1}{L}\PY{l+s+s1}{\PYZsq{}}\PY{p}{,}\PY{l+s+s1}{\PYZsq{}}\PY{l+s+s1}{M}\PY{l+s+s1}{\PYZsq{}}\PY{p}{,}\PY{l+s+s1}{\PYZsq{}}\PY{l+s+s1}{T}\PY{l+s+s1}{\PYZsq{}}\PY{p}{]}\PY{p}{]}
\PY{n}{j2l}\PY{o}{.}\PY{n}{df2table}\PY{p}{(}\PY{n}{DMatCA}\PY{p}{,} \PY{n}{caption}\PY{o}{=}\PY{l+s+s1}{\PYZsq{}}\PY{l+s+s1}{Matriz D \PYZhy{} Viga CA}\PY{l+s+s1}{\PYZsq{}}\PY{p}{)}
\end{Verbatim}
\end{tcolorbox}

    
    \begin{table}[h!]
    \centering
    \caption{Matriz D - Viga CA}
    {\begin{tabular}{lrrr}
\toprule
{} &  L &  M &  T \\
\midrule
M &  2 &  1 & -2 \\
A &  2 &  0 &  0 \\
\textsigma & -1 &  1 & -2 \\
L &  1 &  0 &  0 \\
L &  1 &  0 &  0 \\
\textsigma & -1 &  1 & -2 \\
\bottomrule
\end{tabular}
}
    \label{}
    \end{table}
    

    
    Conforme Rocha (2020) e Carneiro (1993), o teorema dos \(\Pi\)'s de
Vaschy-Buckingham pode produzir \(r=n-k\) números \(\Pi\)'s não
dimensionais, onde \(n\) é o número de variáveis do problema (incluíndo
a variável independente) e \(k\) é o número de grandezas fundamentais
envolvidas (\(L,M,T\)).

Para o problema em questão tem-se \(n=6\) e \(k=3\), logo, 3 números
\(\Pi\)'s podem ser produzidos, onde cada \(\Pi_j\) é um monomio formado
pelas \(n\) variáveis elevadas aos expoentes \(\alpha_{j,n_i}\). A
adimensionalidade de \(\Pi_j\) é garantida montando um vetor
\(\vec\alpha_j\), contendo os expoentes de cada uma das \(n\) variáveis,
e garantindo que \(\vec\alpha_j . {\bf D } = \vec0\). Assim, \(\Pi_j\) e
\(\alpha_j\) são genéricamente dados por:

\begin{equation}
\Pi_j = M^{\alpha_{j,M}} A_s^{\alpha_{j,A_s}} f_y^{\alpha_{j,f_y}} d^{\alpha_{j,d}} b^{\alpha_{j,b}} f_c^{\alpha_{j,f_c}}
\label{Pij}
\end{equation}

\begin{equation}
\vec\alpha_j = [{\alpha_{j,M}}, {\alpha_{j,A_s}}, {\alpha_{j,f_y}}, {\alpha_{j,d}}, {\alpha_{j,b}}, {\alpha_{j,f_c}}]
\label{alphaj}
\end{equation}

    Adotando \(\Pi_1\) como a razão entre as áreas de armadura e concreto
tem-se:

\begin{equation}
\Pi_1 = \frac{A_s}{bd}
\label{Pi1}
\end{equation}

\begin{equation}
\vec\alpha_1 = [0, 1, 0, -1, -1, 0]
\label{alpha1}
\end{equation}

que tem sua adimensionalidade provada por:

    \begin{tcolorbox}[breakable, size=fbox, boxrule=1pt, pad at break*=1mm,colback=cellbackground, colframe=cellborder]
\prompt{In}{incolor}{4}{\boxspacing}
\begin{Verbatim}[commandchars=\\\{\}]
\PY{n}{alpha\PYZus{}Pi1} \PY{o}{=} \PY{n}{np}\PY{o}{.}\PY{n}{array}\PY{p}{(}\PY{p}{[}\PY{l+m+mi}{0}\PY{p}{,} \PY{l+m+mi}{1}\PY{p}{,} \PY{l+m+mi}{0}\PY{p}{,} \PY{o}{\PYZhy{}}\PY{l+m+mi}{1}\PY{p}{,} \PY{o}{\PYZhy{}}\PY{l+m+mi}{1}\PY{p}{,} \PY{l+m+mi}{0}\PY{p}{]}\PY{p}{)}
\PY{n}{prod\PYZus{}Pi1}  \PY{o}{=} \PY{n}{alpha\PYZus{}Pi1}\PY{o}{.}\PY{n}{dot}\PY{p}{(}\PY{n}{DMatCA}\PY{o}{.}\PY{n}{values}\PY{p}{)}
\PY{n+nb}{print}\PY{p}{(}\PY{n}{prod\PYZus{}Pi1}\PY{p}{)}
\end{Verbatim}
\end{tcolorbox}

    \begin{Verbatim}[commandchars=\\\{\}]
[0 0 0]
    \end{Verbatim}

    Para \(\Pi_2\) é adotada a razão entre a tensão de escoamento da
armadura e resistência à compressão do concreto:

\begin{equation}
\Pi_2 = \frac{f_y}{f_c}
\label{Pi2}
\end{equation}

\begin{equation}
\vec\alpha_2 = [0, 0, 1, 0, 0, -1]
\label{alpha2}
\end{equation}

que tem sua adimensionalidade provada por:

    \begin{tcolorbox}[breakable, size=fbox, boxrule=1pt, pad at break*=1mm,colback=cellbackground, colframe=cellborder]
\prompt{In}{incolor}{5}{\boxspacing}
\begin{Verbatim}[commandchars=\\\{\}]
\PY{n}{alpha\PYZus{}Pi2} \PY{o}{=} \PY{n}{np}\PY{o}{.}\PY{n}{array}\PY{p}{(}\PY{p}{[}\PY{l+m+mi}{0}\PY{p}{,} \PY{l+m+mi}{0}\PY{p}{,} \PY{l+m+mi}{1}\PY{p}{,} \PY{l+m+mi}{0}\PY{p}{,} \PY{l+m+mi}{0}\PY{p}{,} \PY{o}{\PYZhy{}}\PY{l+m+mi}{1}\PY{p}{]}\PY{p}{)}
\PY{n}{prod\PYZus{}Pi2}  \PY{o}{=} \PY{n}{alpha\PYZus{}Pi2}\PY{o}{.}\PY{n}{dot}\PY{p}{(}\PY{n}{DMatCA}\PY{o}{.}\PY{n}{values}\PY{p}{)}
\PY{n+nb}{print}\PY{p}{(}\PY{n}{prod\PYZus{}Pi2}\PY{p}{)}
\end{Verbatim}
\end{tcolorbox}

    \begin{Verbatim}[commandchars=\\\{\}]
[0 0 0]
    \end{Verbatim}

    Adotando em \(\Pi_3\) o momento \(M\) como numerador, diferentes
formulações podem ser obtidas como:

\begin{equation}
\Pi_3 = \frac{M}{A_s f_y d}
\label{Pi3a}
\end{equation}

\begin{equation}
\Pi_3 = \frac{M}{bd^2f_c}
\label{Pi3}
\end{equation}

Adotando a segunda formulação, de forma a eliminar de \(\Pi_3\) a
dependencia de \(A_s\), o vetor \(\vec\alpha_3\) é:

\begin{equation}
\vec\alpha_3 = [1, 0, 0, -2, -1, -1]
\label{alpha3}
\end{equation}

que tem sua adimensionalidade provada por:

    \begin{tcolorbox}[breakable, size=fbox, boxrule=1pt, pad at break*=1mm,colback=cellbackground, colframe=cellborder]
\prompt{In}{incolor}{6}{\boxspacing}
\begin{Verbatim}[commandchars=\\\{\}]
\PY{n}{alpha\PYZus{}Pi3} \PY{o}{=} \PY{n}{np}\PY{o}{.}\PY{n}{array}\PY{p}{(}\PY{p}{[}\PY{l+m+mi}{1}\PY{p}{,} \PY{l+m+mi}{0}\PY{p}{,} \PY{l+m+mi}{0}\PY{p}{,} \PY{o}{\PYZhy{}}\PY{l+m+mi}{2}\PY{p}{,} \PY{o}{\PYZhy{}}\PY{l+m+mi}{1}\PY{p}{,} \PY{o}{\PYZhy{}}\PY{l+m+mi}{1}\PY{p}{]}\PY{p}{)}
\PY{n}{prod\PYZus{}Pi3}  \PY{o}{=} \PY{n}{alpha\PYZus{}Pi3}\PY{o}{.}\PY{n}{dot}\PY{p}{(}\PY{n}{DMatCA}\PY{o}{.}\PY{n}{values}\PY{p}{)}
\PY{n+nb}{print}\PY{p}{(}\PY{n}{prod\PYZus{}Pi3}\PY{p}{)}
\end{Verbatim}
\end{tcolorbox}

    \begin{Verbatim}[commandchars=\\\{\}]
[0 0 0]
    \end{Verbatim}

    Dessa forma \ref{expviga} pode ser reescrita em função dos adimensionais
\(\Pi_1\), \(\Pi_2\) e \(\Pi_3\):

\begin{equation}
\Pi_3=\Pi_1\Pi_2-\frac{1}{2}(\Pi_1\Pi_2)^2
\label{F}
\end{equation}

onde observa-se que um único adimensional \(\Pi_{12} = \Pi_1\Pi_2\)
poderia ser empregado, porém este não será adotado.

    Em posse de \ref{F} e de \(\Pi_1\), \(\Pi_2\) e \(\Pi_3\), curvas que
relacionam tais variáveis podem ser traçadas. Visto que \(\Pi_1=A_s/bd\)
é a taxa de armadura tracionada da viga (\(\rho\)), cada valor desse
paramêtro, que pela norma brasileira é limitado a 4\%, será plotado em
uma curva diferente.

Adotando \(\Pi_2\) como eixo horizontal e \(\Pi_3\) como eixo vertical,
considerando \(5 \le \Pi_2 \le 55\), o que engloba combinações de concretos
das classes C20 à C90 e aços CA-50 e CA-60:

    \begin{tcolorbox}[breakable, size=fbox, boxrule=1pt, pad at break*=1mm,colback=cellbackground, colframe=cellborder]
\prompt{In}{incolor}{7}{\boxspacing}
\begin{Verbatim}[commandchars=\\\{\}]
\PY{c+c1}{\PYZsh{} Basedo em Pi1 gera curvas de Pi2 e Pi3 no intervalo [5,55]}
\PY{k}{def} \PY{n+nf}{graf}\PY{p}{(}\PY{n}{Pi1}\PY{p}{)}\PY{p}{:}
    \PY{n}{Pi2} \PY{o}{=} \PY{n}{np}\PY{o}{.}\PY{n}{linspace}\PY{p}{(}\PY{l+m+mi}{5}\PY{p}{,} \PY{l+m+mi}{55}\PY{p}{,} \PY{l+m+mi}{51}\PY{p}{)}
    \PY{n}{Pi3} \PY{o}{=} \PY{n}{Pi1}\PY{o}{*}\PY{n}{Pi2} \PY{o}{\PYZhy{}} \PY{l+m+mi}{1}\PY{o}{/}\PY{l+m+mi}{2}\PY{o}{*}\PY{p}{(}\PY{n}{Pi1}\PY{o}{*}\PY{n}{Pi2}\PY{p}{)}\PY{o}{*}\PY{o}{*}\PY{l+m+mi}{2}
    \PY{n}{plt}\PY{o}{.}\PY{n}{plot}\PY{p}{(}\PY{n}{Pi2}\PY{p}{,} \PY{n}{Pi3}\PY{p}{,} \PY{n}{label}\PY{o}{=}\PY{l+s+sa}{f}\PY{l+s+s1}{\PYZsq{}}\PY{l+s+s1}{\PYZdl{}}\PY{l+s+se}{\PYZbs{}\PYZbs{}}\PY{l+s+s1}{rho=}\PY{l+s+si}{\PYZob{}}\PY{n}{Pi1}\PY{l+s+si}{:}\PY{l+s+s1}{.3f}\PY{l+s+si}{\PYZcb{}}\PY{l+s+s1}{\PYZdl{}}\PY{l+s+s1}{\PYZsq{}}\PY{p}{)}
    
    
\PY{c+c1}{\PYZsh{} Configura plot}
\PY{c+c1}{\PYZsh{} inline ou qt }
\PY{c+c1}{\PYZsh{}\PYZpc{}matplotlib inline}
\PY{n}{matplotlib}\PY{o}{.}\PY{n}{rcParams}\PY{o}{.}\PY{n}{update}\PY{p}{(}\PY{p}{\PYZob{}}\PY{l+s+s1}{\PYZsq{}}\PY{l+s+s1}{font.size}\PY{l+s+s1}{\PYZsq{}}\PY{p}{:} \PY{l+m+mi}{12}\PY{p}{\PYZcb{}}\PY{p}{)}
\PY{n}{plt}\PY{o}{.}\PY{n}{figure}\PY{p}{(}\PY{l+m+mi}{1}\PY{p}{,} \PY{n}{figsize}\PY{o}{=}\PY{p}{(}\PY{l+m+mi}{6}\PY{p}{,}\PY{l+m+mi}{4}\PY{p}{)}\PY{p}{)}
\PY{n}{plt}\PY{o}{.}\PY{n}{grid}\PY{p}{(}\PY{p}{)}
\PY{n}{plt}\PY{o}{.}\PY{n}{axis}\PY{p}{(}\PY{p}{(}\PY{l+m+mi}{5}\PY{p}{,}\PY{l+m+mi}{55}\PY{p}{,}\PY{l+m+mi}{0}\PY{p}{,}\PY{l+m+mf}{0.55}\PY{p}{)}\PY{p}{)}
\PY{n}{plt}\PY{o}{.}\PY{n}{ylabel}\PY{p}{(}\PY{l+s+s1}{\PYZsq{}}\PY{l+s+s1}{\PYZdl{}}\PY{l+s+se}{\PYZbs{}\PYZbs{}}\PY{l+s+s1}{frac }\PY{l+s+si}{\PYZob{}M\PYZcb{}}\PY{l+s+s1}{\PYZob{}}\PY{l+s+s1}{b d\PYZca{}2 f\PYZus{}c\PYZcb{}\PYZdl{}}\PY{l+s+s1}{\PYZsq{}}\PY{p}{,} \PY{n}{size}\PY{o}{=}\PY{l+m+mi}{20}\PY{p}{)}
\PY{n}{plt}\PY{o}{.}\PY{n}{xlabel}\PY{p}{(}\PY{l+s+s1}{\PYZsq{}}\PY{l+s+s1}{\PYZdl{}f\PYZus{}y/f\PYZus{}c\PYZdl{}}\PY{l+s+s1}{\PYZsq{}}\PY{p}{,} \PY{n}{size}\PY{o}{=}\PY{l+m+mi}{15}\PY{p}{)}

\PY{c+c1}{\PYZsh{} Plota linhas para cada rho}
\PY{n}{graf}\PY{p}{(}\PY{l+m+mf}{0.005}\PY{p}{)}
\PY{n}{graf}\PY{p}{(}\PY{l+m+mf}{0.010}\PY{p}{)}
\PY{n}{graf}\PY{p}{(}\PY{l+m+mf}{0.020}\PY{p}{)}
\PY{n}{graf}\PY{p}{(}\PY{l+m+mf}{0.030}\PY{p}{)}
\PY{n}{graf}\PY{p}{(}\PY{l+m+mf}{0.040}\PY{p}{)}

\PY{c+c1}{\PYZsh{} Legenda}
\PY{n}{\PYZus{}} \PY{o}{=} \PY{n}{plt}\PY{o}{.}\PY{n}{legend}\PY{p}{(}\PY{n}{ncol}\PY{o}{=}\PY{l+m+mi}{1}\PY{p}{,} \PY{n}{loc}\PY{o}{=}\PY{l+s+s1}{\PYZsq{}}\PY{l+s+s1}{center left}\PY{l+s+s1}{\PYZsq{}}\PY{p}{,} \PY{n}{bbox\PYZus{}to\PYZus{}anchor}\PY{o}{=}\PY{p}{(}\PY{l+m+mf}{1.0}\PY{p}{,}\PY{l+m+mf}{0.5}\PY{p}{)}\PY{p}{)}
\end{Verbatim}
\end{tcolorbox}

    \begin{center}
    \adjustimage{max size={0.9\linewidth}{0.9\paperheight}}{trab1_files/trab1_21_0.pdf}
    \end{center}
    { \hspace*{\fill} \\}
    
    Através de \ref{expviga} e forçando valores para \(\Pi_1\) (\(\rho\)),
combinações das variáveis de entrada devem produzir pontos sobre as
linhas da figura anterior, dessa forma:

    \begin{tcolorbox}[breakable, size=fbox, boxrule=1pt, pad at break*=1mm,colback=cellbackground, colframe=cellborder]
\prompt{In}{incolor}{8}{\boxspacing}
\begin{Verbatim}[commandchars=\\\{\}]
\PY{c+c1}{\PYZsh{} Pega valores cálcula Mu e Pi\PYZsq{}s}
\PY{k}{def} \PY{n+nf}{PontosValid}\PY{p}{(}\PY{n}{As}\PY{p}{,} \PY{n}{fy}\PY{p}{,} \PY{n}{d}\PY{p}{,} \PY{n}{b}\PY{p}{,} \PY{n}{fc}\PY{p}{)}\PY{p}{:}
    \PY{n}{Mu} \PY{o}{=} \PY{n}{As}\PY{o}{*}\PY{n}{fy}\PY{o}{*}\PY{p}{(}\PY{n}{d}\PY{o}{\PYZhy{}}\PY{n}{As}\PY{o}{*}\PY{n}{fy}\PY{o}{/}\PY{p}{(}\PY{l+m+mi}{2}\PY{o}{*}\PY{n}{b}\PY{o}{*}\PY{n}{fc}\PY{p}{)}\PY{p}{)}
    \PY{n}{Pi1} \PY{o}{=} \PY{n}{As}\PY{o}{/}\PY{p}{(}\PY{n}{b}\PY{o}{*}\PY{n}{d}\PY{p}{)}
    \PY{n}{Pi2} \PY{o}{=} \PY{n}{fy}\PY{o}{/}\PY{n}{fc}
    \PY{n}{Pi3} \PY{o}{=} \PY{n}{Mu}\PY{o}{/}\PY{p}{(}\PY{n}{b}\PY{o}{*}\PY{n}{d}\PY{o}{*}\PY{o}{*}\PY{l+m+mi}{2}\PY{o}{*}\PY{n}{fc}\PY{p}{)}
    
    \PY{c+c1}{\PYZsh{} Cria pontos em grafico do tipo scatter}
    \PY{n}{plt}\PY{o}{.}\PY{n}{scatter}\PY{p}{(}\PY{n}{Pi2}\PY{p}{,} \PY{n}{Pi3}\PY{p}{,} \PY{n}{s}\PY{o}{=}\PY{l+m+mi}{60}\PY{p}{,} \PY{n}{marker}\PY{o}{=}\PY{l+s+s1}{\PYZsq{}}\PY{l+s+s1}{x}\PY{l+s+s1}{\PYZsq{}}\PY{p}{,} \PY{n}{label}\PY{o}{=}\PY{l+s+sa}{f}\PY{l+s+s1}{\PYZsq{}}\PY{l+s+s1}{Valid.: \PYZdl{}}\PY{l+s+se}{\PYZbs{}\PYZbs{}}\PY{l+s+s1}{rho=}\PY{l+s+si}{\PYZob{}}\PY{n}{Pi1}\PY{l+s+si}{:}\PY{l+s+s1}{.3f}\PY{l+s+si}{\PYZcb{}}\PY{l+s+s1}{\PYZdl{}}\PY{l+s+s1}{\PYZsq{}}\PY{p}{)}
    
    
\PY{c+c1}{\PYZsh{} Configura plot}
\PY{n}{plt}\PY{o}{.}\PY{n}{figure}\PY{p}{(}\PY{l+m+mi}{2}\PY{p}{,} \PY{n}{figsize}\PY{o}{=}\PY{p}{(}\PY{l+m+mi}{6}\PY{p}{,}\PY{l+m+mi}{4}\PY{p}{)}\PY{p}{)}
\PY{n}{plt}\PY{o}{.}\PY{n}{grid}\PY{p}{(}\PY{p}{)}
\PY{n}{plt}\PY{o}{.}\PY{n}{axis}\PY{p}{(}\PY{p}{(}\PY{l+m+mi}{5}\PY{p}{,}\PY{l+m+mi}{55}\PY{p}{,}\PY{l+m+mi}{0}\PY{p}{,}\PY{l+m+mf}{0.55}\PY{p}{)}\PY{p}{)}
\PY{n}{plt}\PY{o}{.}\PY{n}{ylabel}\PY{p}{(}\PY{l+s+s1}{\PYZsq{}}\PY{l+s+s1}{\PYZdl{}}\PY{l+s+se}{\PYZbs{}\PYZbs{}}\PY{l+s+s1}{frac }\PY{l+s+si}{\PYZob{}M\PYZcb{}}\PY{l+s+s1}{\PYZob{}}\PY{l+s+s1}{b d\PYZca{}2 f\PYZus{}c\PYZcb{}\PYZdl{}}\PY{l+s+s1}{\PYZsq{}}\PY{p}{,} \PY{n}{size}\PY{o}{=}\PY{l+m+mi}{20}\PY{p}{)}
\PY{n}{plt}\PY{o}{.}\PY{n}{xlabel}\PY{p}{(}\PY{l+s+s1}{\PYZsq{}}\PY{l+s+s1}{\PYZdl{}f\PYZus{}y/f\PYZus{}c\PYZdl{}}\PY{l+s+s1}{\PYZsq{}}\PY{p}{,} \PY{n}{size}\PY{o}{=}\PY{l+m+mi}{15}\PY{p}{)}

\PY{c+c1}{\PYZsh{} Linhas}
\PY{n}{graf}\PY{p}{(}\PY{l+m+mf}{0.005}\PY{p}{)}
\PY{n}{graf}\PY{p}{(}\PY{l+m+mf}{0.010}\PY{p}{)}
\PY{n}{graf}\PY{p}{(}\PY{l+m+mf}{0.020}\PY{p}{)}
\PY{n}{graf}\PY{p}{(}\PY{l+m+mf}{0.030}\PY{p}{)}
\PY{n}{graf}\PY{p}{(}\PY{l+m+mf}{0.040}\PY{p}{)}

\PY{c+c1}{\PYZsh{} Pontos \PYZhy{} Um para cada linha anterior, assim repete as cores}
\PY{n}{PontosValid}\PY{p}{(}\PY{n}{As}\PY{o}{=}\PY{l+m+mf}{0.005}\PY{o}{*}\PY{l+m+mi}{15}\PY{o}{*}\PY{l+m+mi}{36}\PY{p}{,} \PY{n}{fy}\PY{o}{=}\PY{l+m+mi}{50}\PY{o}{/}\PY{l+m+mf}{1.15}\PY{p}{,} \PY{n}{d}\PY{o}{=}\PY{l+m+mi}{36}\PY{p}{,} \PY{n}{b}\PY{o}{=}\PY{l+m+mi}{15}\PY{p}{,} \PY{n}{fc}\PY{o}{=}\PY{l+m+mi}{2}\PY{o}{*}\PY{l+m+mf}{0.85}\PY{o}{/}\PY{l+m+mf}{1.4}\PY{p}{)}
\PY{n}{PontosValid}\PY{p}{(}\PY{n}{As}\PY{o}{=}\PY{l+m+mf}{0.01}\PY{o}{*}\PY{l+m+mi}{15}\PY{o}{*}\PY{l+m+mi}{60}\PY{p}{,} \PY{n}{fy}\PY{o}{=}\PY{l+m+mi}{50}\PY{o}{/}\PY{l+m+mf}{1.15}\PY{p}{,} \PY{n}{d}\PY{o}{=}\PY{l+m+mi}{60}\PY{p}{,} \PY{n}{b}\PY{o}{=}\PY{l+m+mi}{15}\PY{p}{,} \PY{n}{fc}\PY{o}{=}\PY{l+m+mi}{4}\PY{o}{*}\PY{l+m+mf}{0.85}\PY{o}{/}\PY{l+m+mf}{1.4}\PY{p}{)}
\PY{n}{PontosValid}\PY{p}{(}\PY{n}{As}\PY{o}{=}\PY{l+m+mf}{0.02}\PY{o}{*}\PY{l+m+mi}{10}\PY{o}{*}\PY{l+m+mi}{36}\PY{p}{,} \PY{n}{fy}\PY{o}{=}\PY{l+m+mi}{40}\PY{o}{/}\PY{l+m+mf}{1.15}\PY{p}{,} \PY{n}{d}\PY{o}{=}\PY{l+m+mi}{36}\PY{p}{,} \PY{n}{b}\PY{o}{=}\PY{l+m+mi}{10}\PY{p}{,} \PY{n}{fc}\PY{o}{=}\PY{l+m+mi}{5}\PY{p}{)}
\PY{n}{PontosValid}\PY{p}{(}\PY{n}{As}\PY{o}{=}\PY{l+m+mf}{0.03}\PY{o}{*}\PY{l+m+mi}{15}\PY{o}{*}\PY{l+m+mi}{36}\PY{p}{,} \PY{n}{fy}\PY{o}{=}\PY{l+m+mi}{50}\PY{o}{/}\PY{l+m+mf}{1.15}\PY{p}{,} \PY{n}{d}\PY{o}{=}\PY{l+m+mi}{36}\PY{p}{,} \PY{n}{b}\PY{o}{=}\PY{l+m+mi}{15}\PY{p}{,} \PY{n}{fc}\PY{o}{=}\PY{l+m+mi}{3}\PY{o}{*}\PY{l+m+mf}{1.14}\PY{p}{)}
\PY{n}{PontosValid}\PY{p}{(}\PY{n}{As}\PY{o}{=}\PY{l+m+mf}{0.04}\PY{o}{*}\PY{l+m+mi}{15}\PY{o}{*}\PY{l+m+mi}{36}\PY{p}{,} \PY{n}{fy}\PY{o}{=}\PY{l+m+mi}{60}\PY{o}{/}\PY{l+m+mf}{1.15}\PY{p}{,} \PY{n}{d}\PY{o}{=}\PY{l+m+mi}{36}\PY{p}{,} \PY{n}{b}\PY{o}{=}\PY{l+m+mi}{15}\PY{p}{,} \PY{n}{fc}\PY{o}{=}\PY{l+m+mi}{2}\PY{p}{)}

\PY{c+c1}{\PYZsh{} Legenda}
\PY{n}{\PYZus{}} \PY{o}{=} \PY{n}{plt}\PY{o}{.}\PY{n}{legend}\PY{p}{(}\PY{n}{ncol}\PY{o}{=}\PY{l+m+mi}{1}\PY{p}{,} \PY{n}{loc}\PY{o}{=}\PY{l+s+s1}{\PYZsq{}}\PY{l+s+s1}{center left}\PY{l+s+s1}{\PYZsq{}}\PY{p}{,} \PY{n}{bbox\PYZus{}to\PYZus{}anchor}\PY{o}{=}\PY{p}{(}\PY{l+m+mf}{1.0}\PY{p}{,}\PY{l+m+mf}{0.5}\PY{p}{)}\PY{p}{)}
\end{Verbatim}
\end{tcolorbox}

    \begin{center}
    \adjustimage{max size={0.9\linewidth}{0.9\paperheight}}{trab1_files/trab1_23_0.pdf}
    \end{center}
    { \hspace*{\fill} \\}
    
    Assim, pode-se concluir que a adimensionalização e figura gerada estão
corretas. Cabe observar que sua aplicação prática requer a verificação
do domínio da peça pela norma vigente, visto que a expressão
\ref{expviga} considera o escoamento da armadura.

    \hypertarget{referuxeancias-bibliogruxe1ficas}{%
\section{Referências Bibliográficas}\label{referuxeancias-bibliogruxe1ficas}}

Carneiro, F. L. {\bf Análise dimensional e teoria da semelhança e dos modelos
físicos}. Rio de Janeiro: Editora UFRJ. 1993.

Rocha, M. M. {\bf PEC00144 - Experimental Methods in Civil Engineering}. 2020.
Disponível em: \url{https://github.com/mmaiarocha/PEC00144}.

White, F. M. {\bf Fluid Mechanics}. 8th ed. McGraw-Hill Education. 2017.


    
    
    
\end{document}
