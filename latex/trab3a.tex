C:/Programacao/_MinhasBibliotecas/Jupyter2Latex/latex-template/article-base.tex

\author{Eduardo Pagnussat Titello}
\title{PEC00144 - Métodos Experimentais em Engenharia Civil}
\subtitle{Trabalho 3A}
\date{Dezembro de 2020}


\begin{document}
	
	\maketitle


Este trabalho tem por objetivo: avaliar através de simulações de Monte
Carlo a propagação de erro no modelo empregado no trabalho anterior,
adotando variáveis Gaussianas.

    \begin{tcolorbox}[breakable, size=fbox, boxrule=1pt, pad at break*=1mm,colback=cellbackground, colframe=cellborder]
\prompt{In}{incolor}{2}{\boxspacing}
\begin{Verbatim}[commandchars=\\\{\}]
\PY{c+c1}{\PYZsh{} Importando e configurando módulos}
\PY{k+kn}{import} \PY{n+nn}{numpy} \PY{k}{as} \PY{n+nn}{np}
\PY{k+kn}{import} \PY{n+nn}{pandas} \PY{k}{as} \PY{n+nn}{pd} 
\PY{k+kn}{import} \PY{n+nn}{matplotlib}\PY{n+nn}{.}\PY{n+nn}{pyplot} \PY{k}{as} \PY{n+nn}{plt}
\PY{o}{\PYZpc{}}\PY{k}{config} InlineBackend.figure\PYZus{}format = \PYZsq{}svg\PYZsq{} \PYZsh{} Muda backend do jupyter para SVG ;)
\PY{k+kn}{import} \PY{n+nn}{jupyter2latex} \PY{k}{as} \PY{n+nn}{j2l} \PY{c+c1}{\PYZsh{} Uma maneira que encontrei para tabelas ficarem ok (github.com/dutitello/Jupyter2Latex)}
\PY{k+kn}{from} \PY{n+nn}{numpy} \PY{k+kn}{import} \PY{n}{pi}
\PY{k+kn}{import} \PY{n+nn}{scipy}\PY{n+nn}{.}\PY{n+nn}{stats} \PY{k}{as} \PY{n+nn}{st}
\end{Verbatim}
\end{tcolorbox}

    \hypertarget{recapitulauxe7uxe3o-do-problema-adotado}{%
\section{Recapitulação do problema
adotado}\label{recapitulauxe7uxe3o-do-problema-adotado}}

O problema adotado consiste na avaliação das duas primeiras frequências
naturais de vibração de um poste em concreto armado, considerando seu
comportamento como elástico linear. O poste tem comprimento efetivo
\(L=12m\), seção constante de diâmetro externo \(d_e=50cm\) e espessura
\(e=6cm\).

O modelo reduzido empregado para avaliação do problema consiste em um
cilindro de alumínio de diâmetro \(d_e=6mm\), massa linear
\(\rho=2700kg/m^3\) e módulo de elastícidade \(E=70GPa\). Adotando como
escala de comprimento \(1:25\) (modelo reduzido tem comprimento
\(L=48cm\)) e através das propriedades do perfil empregado, os três
fatores de escala impostos são:

\begin{itemize}
\item
  Comprimento - \(L = 1:25\)
\item
  Massa linear - \(\mu_L = 1:2716\)
\item
  Rígidez à flexão - \(EI = 1:1.285\times10^7\)
\end{itemize}

Através da análise dimensional realizada anteriormente, o fator de
escala de frequências obtido foi \(1:0.110\) ou \(\lambda_f = 9.085\).

    As frequências naturais de vibração do modelo são dadas por \ref{eq:wn}:

\begin{equation}
f_n = \frac{1}{2\pi} {\Big( \frac{\alpha_n}{L} \Big)}^2 \sqrt{\frac{EI}{\mu_L}}
\label{eq:wn}
\end{equation}

onde, para as duas primeiras frequências naturais, \(\alpha_1=1.88\) e
\(\alpha_2=4.69\), respectivamente. A área da seção transversal circular
\(A\) e seu momento de inércia \(I\) são dados por:

\begin{equation}
A = \frac{\pi d_e^2}{4}
\label{eq:Ac}
\end{equation}

\begin{equation}
I = \frac{\pi d_e^4}{64}
\label{eq:Ic}
\end{equation}

    \hypertarget{propriedades-estatuxedsticas-do-modelo-reduzido}{%
\section{Propriedades estatísticas do modelo
reduzido}\label{propriedades-estatuxedsticas-do-modelo-reduzido}}

Desprezando possíveis erros de matemáticos de modelagem, conforme
\ref{eq:wn}, \ref{eq:Ac} e \ref{eq:Ic}, os parâmetros que podem
introduzir erros no modelo reduzido são \(E\), \(d_e\), \(\rho\) e
\(L\). Dado o grande controle tecnológico existente hoje nas industrias,
para as propriedades \(E\), \(d_e\) e \(\rho\) são adotados coeficientes
de variação de \(5\%\), enquanto, para \(L\), visto que os perfis devem
ser cortados pelo usuário final, é fixado um desvio padrão de \(5mm\).
Assim, as variáveis aleatórias e suas propriedades são:

\begin{longtable}[]{@{}lllll@{}}
\toprule
\begin{minipage}[b]{0.21\columnwidth}\raggedright
Var. aleatória\strut
\end{minipage} & \begin{minipage}[b]{0.08\columnwidth}\raggedright
Símbolo\strut
\end{minipage} & \begin{minipage}[b]{0.13\columnwidth}\raggedright
Média (\(\mu\))\strut
\end{minipage} & \begin{minipage}[b]{0.22\columnwidth}\raggedright
Coef. de Variação (\(CV\))\strut
\end{minipage} & \begin{minipage}[b]{0.22\columnwidth}\raggedright
Desvio Padrão (\(\sigma\))\strut
\end{minipage}\tabularnewline
\midrule
\endhead
\begin{minipage}[t]{0.21\columnwidth}\raggedright
Módulo de elasticidade\strut
\end{minipage} & \begin{minipage}[t]{0.08\columnwidth}\raggedright
\(E\)\strut
\end{minipage} & \begin{minipage}[t]{0.13\columnwidth}\raggedright
70 GPa\strut
\end{minipage} & \begin{minipage}[t]{0.22\columnwidth}\raggedright
5\%\strut
\end{minipage} & \begin{minipage}[t]{0.22\columnwidth}\raggedright
3.5 GPa\strut
\end{minipage}\tabularnewline
\begin{minipage}[t]{0.21\columnwidth}\raggedright
Diâmetro externo\strut
\end{minipage} & \begin{minipage}[t]{0.08\columnwidth}\raggedright
\(d_e\)\strut
\end{minipage} & \begin{minipage}[t]{0.13\columnwidth}\raggedright
6 mm\strut
\end{minipage} & \begin{minipage}[t]{0.22\columnwidth}\raggedright
5\%\strut
\end{minipage} & \begin{minipage}[t]{0.22\columnwidth}\raggedright
0.3 mm\strut
\end{minipage}\tabularnewline
\begin{minipage}[t]{0.21\columnwidth}\raggedright
Massa específica\strut
\end{minipage} & \begin{minipage}[t]{0.08\columnwidth}\raggedright
\(\rho\)\strut
\end{minipage} & \begin{minipage}[t]{0.13\columnwidth}\raggedright
2700 kg/m³\strut
\end{minipage} & \begin{minipage}[t]{0.22\columnwidth}\raggedright
5\%\strut
\end{minipage} & \begin{minipage}[t]{0.22\columnwidth}\raggedright
135 kg/m³\strut
\end{minipage}\tabularnewline
\begin{minipage}[t]{0.21\columnwidth}\raggedright
Comprimento\strut
\end{minipage} & \begin{minipage}[t]{0.08\columnwidth}\raggedright
\(L\)\strut
\end{minipage} & \begin{minipage}[t]{0.13\columnwidth}\raggedright
48 cm\strut
\end{minipage} & \begin{minipage}[t]{0.22\columnwidth}\raggedright
-\strut
\end{minipage} & \begin{minipage}[t]{0.22\columnwidth}\raggedright
0.5 cm\strut
\end{minipage}\tabularnewline
\bottomrule
\end{longtable}

Com isso as distribuições podem ser construídas no SciPy.

    \begin{tcolorbox}[breakable, size=fbox, boxrule=1pt, pad at break*=1mm,colback=cellbackground, colframe=cellborder]
\prompt{In}{incolor}{3}{\boxspacing}
\begin{Verbatim}[commandchars=\\\{\}]
\PY{c+c1}{\PYZsh{} Módulo de elasticidade em N/m²}
\PY{n}{mu\PYZus{}E}   \PY{o}{=} \PY{l+m+mf}{70E9} 
\PY{n}{std\PYZus{}E}  \PY{o}{=} \PY{l+m+mf}{0.05}\PY{o}{*}\PY{n}{mu\PYZus{}E}
\PY{n}{rv\PYZus{}E}   \PY{o}{=} \PY{n}{st}\PY{o}{.}\PY{n}{norm}\PY{p}{(}\PY{n}{mu\PYZus{}E}\PY{p}{,} \PY{n}{std\PYZus{}E}\PY{p}{)}    

\PY{c+c1}{\PYZsh{} Diametro do perfil em m}
\PY{n}{mu\PYZus{}de}  \PY{o}{=} \PY{l+m+mi}{6}\PY{o}{/}\PY{l+m+mi}{1000}
\PY{n}{std\PYZus{}de} \PY{o}{=} \PY{l+m+mf}{0.05}\PY{o}{*}\PY{n}{mu\PYZus{}de}
\PY{n}{rv\PYZus{}de}  \PY{o}{=} \PY{n}{st}\PY{o}{.}\PY{n}{norm}\PY{p}{(}\PY{n}{mu\PYZus{}de}\PY{p}{,} \PY{n}{std\PYZus{}de}\PY{p}{)}

\PY{c+c1}{\PYZsh{} Massa específica em kg/m³}
\PY{n}{mu\PYZus{}rho}  \PY{o}{=} \PY{l+m+mi}{2700}
\PY{n}{std\PYZus{}rho} \PY{o}{=} \PY{l+m+mf}{0.05}\PY{o}{*}\PY{n}{mu\PYZus{}rho}
\PY{n}{rv\PYZus{}rho}  \PY{o}{=} \PY{n}{st}\PY{o}{.}\PY{n}{norm}\PY{p}{(}\PY{n}{mu\PYZus{}rho}\PY{p}{,} \PY{n}{std\PYZus{}rho}\PY{p}{)}

\PY{c+c1}{\PYZsh{} Comprimento em m}
\PY{n}{mu\PYZus{}L}    \PY{o}{=} \PY{l+m+mi}{48}\PY{o}{/}\PY{l+m+mi}{100}
\PY{n}{std\PYZus{}L}   \PY{o}{=} \PY{l+m+mf}{0.5}\PY{o}{/}\PY{l+m+mi}{100}
\PY{n}{rv\PYZus{}L}    \PY{o}{=} \PY{n}{st}\PY{o}{.}\PY{n}{norm}\PY{p}{(}\PY{n}{mu\PYZus{}L}\PY{p}{,} \PY{n}{std\PYZus{}L}\PY{p}{)} 
\end{Verbatim}
\end{tcolorbox}

    \hypertarget{funuxe7uxe3o-para-avaliauxe7uxe3o-do-modelo}{%
\section{Função para avaliação do
modelo}\label{funuxe7uxe3o-para-avaliauxe7uxe3o-do-modelo}}

Para aplicação do método de simulação por Monte Carlo a equação
\ref{eq:wn} é introduzida em uma função dependente apenas das variáveis
aleatórias de entrada consideradas:

    \begin{tcolorbox}[breakable, size=fbox, boxrule=1pt, pad at break*=1mm,colback=cellbackground, colframe=cellborder]
\prompt{In}{incolor}{4}{\boxspacing}
\begin{Verbatim}[commandchars=\\\{\}]
\PY{k}{def} \PY{n+nf}{frequencias}\PY{p}{(}\PY{n}{E}\PY{p}{,} \PY{n}{de}\PY{p}{,} \PY{n}{rho}\PY{p}{,} \PY{n}{L}\PY{p}{)}\PY{p}{:}
    \PY{n}{A}   \PY{o}{=} \PY{l+m+mi}{1}\PY{o}{/}\PY{l+m+mi}{4}\PY{o}{*}\PY{n}{pi}\PY{o}{*}\PY{n}{de}\PY{o}{*}\PY{o}{*}\PY{l+m+mi}{2} 
    \PY{n}{I}   \PY{o}{=} \PY{l+m+mi}{1}\PY{o}{/}\PY{l+m+mi}{64}\PY{o}{*}\PY{n}{pi}\PY{o}{*}\PY{n}{de}\PY{o}{*}\PY{o}{*}\PY{l+m+mi}{4}
    \PY{n}{muL} \PY{o}{=} \PY{n}{A}\PY{o}{*}\PY{n}{rho}
    
    \PY{c+c1}{\PYZsh{} Frequencias naturais do modelo}
    \PY{n}{fns} \PY{o}{=} \PY{l+m+mi}{1}\PY{o}{/}\PY{p}{(}\PY{l+m+mi}{2}\PY{o}{*}\PY{n}{pi}\PY{p}{)} \PY{o}{*} \PY{l+m+mi}{1}\PY{o}{/}\PY{n}{L}\PY{o}{*}\PY{o}{*}\PY{l+m+mi}{2} \PY{o}{*} \PY{p}{(}\PY{n}{E}\PY{o}{*}\PY{n}{I}\PY{o}{/}\PY{n}{muL}\PY{p}{)}\PY{o}{*}\PY{o}{*}\PY{l+m+mf}{0.5}
    \PY{n}{fn1\PYZus{}m} \PY{o}{=} \PY{l+m+mf}{1.88}\PY{o}{*}\PY{o}{*}\PY{l+m+mi}{2} \PY{o}{*} \PY{n}{fns}
    \PY{n}{fn2\PYZus{}m} \PY{o}{=} \PY{l+m+mf}{4.69}\PY{o}{*}\PY{o}{*}\PY{l+m+mi}{2} \PY{o}{*} \PY{n}{fns}
    
    \PY{c+c1}{\PYZsh{} Frequencias naturais da estrutura}
    \PY{n}{fn1\PYZus{}e} \PY{o}{=} \PY{n}{fn1\PYZus{}m} \PY{o}{*} \PY{l+m+mf}{1.100726e\PYZhy{}01}
    \PY{n}{fn2\PYZus{}e} \PY{o}{=} \PY{n}{fn2\PYZus{}m} \PY{o}{*} \PY{l+m+mf}{1.100726e\PYZhy{}01}
    
    \PY{k}{return} \PY{n}{fn1\PYZus{}m}\PY{p}{,} \PY{n}{fn2\PYZus{}m}\PY{p}{,} \PY{n}{fn1\PYZus{}e}\PY{p}{,} \PY{n}{fn2\PYZus{}e}
\end{Verbatim}
\end{tcolorbox}

    Aplicando sobre a função os valores médios devem ser obtidas as
frequências observadas no trabalho anterior:

\begin{itemize}
\item
  Para o modelo reduzido: 18.647 Hz e 116.049 Hz
\item
  Para a estrutura real: 2.053 Hz e 12.774 Hz
\end{itemize}

    \begin{tcolorbox}[breakable, size=fbox, boxrule=1pt, pad at break*=1mm,colback=cellbackground, colframe=cellborder]
\prompt{In}{incolor}{5}{\boxspacing}
\begin{Verbatim}[commandchars=\\\{\}]
\PY{n}{freqs} \PY{o}{=} \PY{n}{frequencias}\PY{p}{(}\PY{n}{mu\PYZus{}E}\PY{p}{,} \PY{n}{mu\PYZus{}de}\PY{p}{,} \PY{n}{mu\PYZus{}rho}\PY{p}{,} \PY{n}{mu\PYZus{}L}\PY{p}{)}
\PY{n+nb}{print}\PY{p}{(}\PY{n}{freqs}\PY{p}{)}
\end{Verbatim}
\end{tcolorbox}

    \begin{Verbatim}[commandchars=\\\{\}]
(18.647119431070642, 116.04909000616033, 2.0525369182884665, 12.773825064612083)
    \end{Verbatim}

    Ok, a função retorna os valores esperados.

    \hypertarget{avaliauxe7uxe3o-da-propagauxe7uxe3o-de-erro}{%
\section{Avaliação da propagação de
erro}\label{avaliauxe7uxe3o-da-propagauxe7uxe3o-de-erro}}

Para avaliação da propagação de erro, dado o baixo custo computacional
do problema, são realizadas \(5 \times 10^6\) simulações e os resultados
são avaliados cumulativamente, de forma a observar a convergência do
processo.

    \begin{tcolorbox}[breakable, size=fbox, boxrule=1pt, pad at break*=1mm,colback=cellbackground, colframe=cellborder]
\prompt{In}{incolor}{6}{\boxspacing}
\begin{Verbatim}[commandchars=\\\{\}]
\PY{c+c1}{\PYZsh{} Controles}
\PY{n}{N}    \PY{o}{=} \PY{n+nb}{int}\PY{p}{(}\PY{l+m+mf}{5E6}\PY{p}{)} \PY{c+c1}{\PYZsh{} Total de simulações}
\PY{n}{minp} \PY{o}{=} \PY{l+m+mi}{10}       \PY{c+c1}{\PYZsh{} Mínimo de pontos para cálcular média e desvio padrão}
\PY{n}{np}\PY{o}{.}\PY{n}{random}\PY{o}{.}\PY{n}{seed}\PY{p}{(}\PY{l+m+mi}{666}\PY{p}{)}

\PY{c+c1}{\PYZsh{} Construção de DataFrame para armazenar os dados}
\PY{n}{dados} \PY{o}{=} \PY{n}{pd}\PY{o}{.}\PY{n}{DataFrame}\PY{p}{(}\PY{p}{)}

\PY{c+c1}{\PYZsh{} Geração de dados aleatórios}
\PY{n}{dados}\PY{p}{[}\PY{l+s+s1}{\PYZsq{}}\PY{l+s+s1}{E}\PY{l+s+s1}{\PYZsq{}}\PY{p}{]}   \PY{o}{=} \PY{n}{rv\PYZus{}E}\PY{o}{.}\PY{n}{rvs}\PY{p}{(}\PY{n}{N}\PY{p}{)}
\PY{n}{dados}\PY{p}{[}\PY{l+s+s1}{\PYZsq{}}\PY{l+s+s1}{de}\PY{l+s+s1}{\PYZsq{}}\PY{p}{]}  \PY{o}{=} \PY{n}{rv\PYZus{}de}\PY{o}{.}\PY{n}{rvs}\PY{p}{(}\PY{n}{N}\PY{p}{)}
\PY{n}{dados}\PY{p}{[}\PY{l+s+s1}{\PYZsq{}}\PY{l+s+s1}{rho}\PY{l+s+s1}{\PYZsq{}}\PY{p}{]} \PY{o}{=} \PY{n}{rv\PYZus{}rho}\PY{o}{.}\PY{n}{rvs}\PY{p}{(}\PY{n}{N}\PY{p}{)}
\PY{n}{dados}\PY{p}{[}\PY{l+s+s1}{\PYZsq{}}\PY{l+s+s1}{L}\PY{l+s+s1}{\PYZsq{}}\PY{p}{]}   \PY{o}{=} \PY{n}{rv\PYZus{}L}\PY{o}{.}\PY{n}{rvs}\PY{p}{(}\PY{n}{N}\PY{p}{)}

\PY{c+c1}{\PYZsh{} Cálculo das frequencias naturais}
\PY{n}{dados}\PY{p}{[}\PY{l+s+s1}{\PYZsq{}}\PY{l+s+s1}{fn1\PYZus{}m}\PY{l+s+s1}{\PYZsq{}}\PY{p}{]}\PY{p}{,} \PY{n}{dados}\PY{p}{[}\PY{l+s+s1}{\PYZsq{}}\PY{l+s+s1}{fn2\PYZus{}m}\PY{l+s+s1}{\PYZsq{}}\PY{p}{]}\PY{p}{,} \PY{n}{dados}\PY{p}{[}\PY{l+s+s1}{\PYZsq{}}\PY{l+s+s1}{fn1\PYZus{}e}\PY{l+s+s1}{\PYZsq{}}\PY{p}{]}\PY{p}{,} \PY{n}{dados}\PY{p}{[}\PY{l+s+s1}{\PYZsq{}}\PY{l+s+s1}{fn2\PYZus{}e}\PY{l+s+s1}{\PYZsq{}}\PY{p}{]} \PY{o}{=} \PY{n}{frequencias}\PY{p}{(}\PY{n}{dados}\PY{p}{[}\PY{l+s+s1}{\PYZsq{}}\PY{l+s+s1}{E}\PY{l+s+s1}{\PYZsq{}}\PY{p}{]}\PY{o}{.}\PY{n}{values}\PY{p}{,} \PY{n}{dados}\PY{p}{[}\PY{l+s+s1}{\PYZsq{}}\PY{l+s+s1}{de}\PY{l+s+s1}{\PYZsq{}}\PY{p}{]}\PY{o}{.}\PY{n}{values}\PY{p}{,} \PY{n}{dados}\PY{p}{[}\PY{l+s+s1}{\PYZsq{}}\PY{l+s+s1}{rho}\PY{l+s+s1}{\PYZsq{}}\PY{p}{]}\PY{o}{.}\PY{n}{values}\PY{p}{,} \PY{n}{dados}\PY{p}{[}\PY{l+s+s1}{\PYZsq{}}\PY{l+s+s1}{L}\PY{l+s+s1}{\PYZsq{}}\PY{p}{]}\PY{o}{.}\PY{n}{values}\PY{p}{)}
\end{Verbatim}
\end{tcolorbox}

    Uma descrição geral dos dados é apresentada nas Tabelas \ref{tab:descE}
e \ref{tab:descf}, onde as médias e o desvios padrões dos paramêtros de
entrada convergem para os valores fornecidos, enquanto as médias das
frequências naturais apresentam pequenas diferenças em relação aos
valores já conhecidos.

    \begin{tcolorbox}[breakable, size=fbox, boxrule=1pt, pad at break*=1mm,colback=cellbackground, colframe=cellborder]
\prompt{In}{incolor}{7}{\boxspacing}
\begin{Verbatim}[commandchars=\\\{\}]
\PY{n}{pd}\PY{o}{.}\PY{n}{set\PYZus{}option}\PY{p}{(}\PY{l+s+s1}{\PYZsq{}}\PY{l+s+s1}{display.float\PYZus{}format}\PY{l+s+s1}{\PYZsq{}}\PY{p}{,} \PY{k}{lambda} \PY{n}{x}\PY{p}{:} \PY{l+s+sa}{f}\PY{l+s+s1}{\PYZsq{}}\PY{l+s+si}{\PYZob{}}\PY{n}{x}\PY{l+s+si}{:}\PY{l+s+s1}{.3E}\PY{l+s+si}{\PYZcb{}}\PY{l+s+s1}{\PYZsq{}}\PY{p}{)}

\PY{n}{desc\PYZus{}geral} \PY{o}{=} \PY{n}{dados}\PY{o}{.}\PY{n}{describe}\PY{p}{(}\PY{p}{)}\PY{o}{.}\PY{n}{loc}\PY{p}{[}\PY{p}{[}\PY{l+s+s1}{\PYZsq{}}\PY{l+s+s1}{mean}\PY{l+s+s1}{\PYZsq{}}\PY{p}{,} \PY{l+s+s1}{\PYZsq{}}\PY{l+s+s1}{std}\PY{l+s+s1}{\PYZsq{}}\PY{p}{,} \PY{l+s+s1}{\PYZsq{}}\PY{l+s+s1}{min}\PY{l+s+s1}{\PYZsq{}}\PY{p}{,} \PY{l+s+s1}{\PYZsq{}}\PY{l+s+s1}{max}\PY{l+s+s1}{\PYZsq{}}\PY{p}{]}\PY{p}{]}
\PY{n}{j2l}\PY{o}{.}\PY{n}{df2table}\PY{p}{(}\PY{n}{desc\PYZus{}geral}\PY{p}{[}\PY{p}{[}\PY{l+s+s1}{\PYZsq{}}\PY{l+s+s1}{E}\PY{l+s+s1}{\PYZsq{}}\PY{p}{,} \PY{l+s+s1}{\PYZsq{}}\PY{l+s+s1}{de}\PY{l+s+s1}{\PYZsq{}}\PY{p}{,} \PY{l+s+s1}{\PYZsq{}}\PY{l+s+s1}{rho}\PY{l+s+s1}{\PYZsq{}}\PY{p}{,} \PY{l+s+s1}{\PYZsq{}}\PY{l+s+s1}{L}\PY{l+s+s1}{\PYZsq{}}\PY{p}{]}\PY{p}{]}\PY{p}{,} \PY{l+s+s1}{\PYZsq{}}\PY{l+s+s1}{Descrição geral dos dados de entrada}\PY{l+s+s1}{\PYZsq{}}\PY{p}{,} \PY{l+s+s1}{\PYZsq{}}\PY{l+s+s1}{tab:descE}\PY{l+s+s1}{\PYZsq{}}\PY{p}{)}
\PY{n}{j2l}\PY{o}{.}\PY{n}{df2table}\PY{p}{(}\PY{n}{desc\PYZus{}geral}\PY{p}{[}\PY{p}{[}\PY{l+s+s1}{\PYZsq{}}\PY{l+s+s1}{fn1\PYZus{}m}\PY{l+s+s1}{\PYZsq{}}\PY{p}{,} \PY{l+s+s1}{\PYZsq{}}\PY{l+s+s1}{fn2\PYZus{}m}\PY{l+s+s1}{\PYZsq{}}\PY{p}{,} \PY{l+s+s1}{\PYZsq{}}\PY{l+s+s1}{fn1\PYZus{}e}\PY{l+s+s1}{\PYZsq{}}\PY{p}{,} \PY{l+s+s1}{\PYZsq{}}\PY{l+s+s1}{fn2\PYZus{}e}\PY{l+s+s1}{\PYZsq{}}\PY{p}{]}\PY{p}{]}\PY{p}{,} \PY{l+s+s1}{\PYZsq{}}\PY{l+s+s1}{Descrição geral das frequências naturais}\PY{l+s+s1}{\PYZsq{}}\PY{p}{,} \PY{l+s+s1}{\PYZsq{}}\PY{l+s+s1}{tab:descf}\PY{l+s+s1}{\PYZsq{}}\PY{p}{)}
\end{Verbatim}
\end{tcolorbox}

    
    \begin{table}[H]
    \centering
    \caption{Descrição geral dos dados de entrada}
    {\begin{tabular}{lrrrr}
\toprule
{} &         E &        de &       rho &         L \\
\midrule
mean & 7.000E+10 & 6.000E-03 & 2.700E+03 & 4.800E-01 \\
std  & 3.500E+09 & 2.999E-04 & 1.350E+02 & 4.999E-03 \\
min  & 5.228E+10 & 4.568E-03 & 1.985E+03 & 4.558E-01 \\
max  & 8.725E+10 & 7.508E-03 & 3.394E+03 & 5.057E-01 \\
\bottomrule
\end{tabular}
}
    \label{tab:descE}
    \end{table}
    

    
    
    \begin{table}[H]
    \centering
    \caption{Descrição geral das frequências naturais}
    {\begin{tabular}{lrrrr}
\toprule
{} &     fn1\_m &     fn2\_m &     fn1\_e &     fn2\_e \\
\midrule
mean & 1.867E+01 & 1.162E+02 & 2.055E+00 & 1.279E+01 \\
std  & 1.209E+00 & 7.524E+00 & 1.331E-01 & 8.282E-01 \\
min  & 1.306E+01 & 8.127E+01 & 1.437E+00 & 8.946E+00 \\
max  & 2.573E+01 & 1.601E+02 & 2.832E+00 & 1.763E+01 \\
\bottomrule
\end{tabular}
}
    \label{tab:descf}
    \end{table}
    

    
    Nos resultados pode ainda ser observada a linearidade da transformação
de escala do modelo, onde:

\begin{equation}
\mu_{f,Est. Real} = \mu_{f, Mod. Red.} \times 1/\lambda_f
\end{equation}

\begin{equation}
\sigma_{f, Est. Real} = \sigma_{f, Mod. Red.} \times 1/\lambda_f
\end{equation}

    Dessa forma, as análises a seguir são realizas apenas em relação ao
modelo reduzido, objeto principal do estudo.

Através da análise estatística cumulativa dos resultados obtidos pelo
método de Monte Carlo a convergência do processo com o aumento do número
de simulações pode ser determinada:

    \begin{tcolorbox}[breakable, size=fbox, boxrule=1pt, pad at break*=1mm,colback=cellbackground, colframe=cellborder]
\prompt{In}{incolor}{8}{\boxspacing}
\begin{Verbatim}[commandchars=\\\{\}]
\PY{n}{roll}  \PY{o}{=} \PY{n}{dados}\PY{p}{[}\PY{p}{[}\PY{l+s+s1}{\PYZsq{}}\PY{l+s+s1}{fn1\PYZus{}m}\PY{l+s+s1}{\PYZsq{}}\PY{p}{,} \PY{l+s+s1}{\PYZsq{}}\PY{l+s+s1}{fn2\PYZus{}m}\PY{l+s+s1}{\PYZsq{}}\PY{p}{]}\PY{p}{]}\PY{o}{.}\PY{n}{rolling}\PY{p}{(}\PY{n}{N}\PY{p}{,} \PY{n}{min\PYZus{}periods}\PY{o}{=}\PY{n}{minp}\PY{p}{)}
\PY{n}{stds}  \PY{o}{=} \PY{n}{roll}\PY{o}{.}\PY{n}{std}\PY{p}{(}\PY{p}{)}
\PY{n}{means} \PY{o}{=} \PY{n}{roll}\PY{o}{.}\PY{n}{mean}\PY{p}{(}\PY{p}{)}
\end{Verbatim}
\end{tcolorbox}

    \begin{tcolorbox}[breakable, size=fbox, boxrule=1pt, pad at break*=1mm,colback=cellbackground, colframe=cellborder]
\prompt{In}{incolor}{9}{\boxspacing}
\begin{Verbatim}[commandchars=\\\{\}]
\PY{n}{figMeans} \PY{o}{=} \PY{n}{plt}\PY{o}{.}\PY{n}{figure}\PY{p}{(}\PY{l+m+mi}{1}\PY{p}{,} \PY{n}{figsize}\PY{o}{=}\PY{p}{(}\PY{l+m+mi}{10}\PY{p}{,}\PY{l+m+mi}{6}\PY{p}{)}\PY{p}{)}

\PY{n}{pl}  \PY{o}{=} \PY{n}{plt}\PY{o}{.}\PY{n}{subplot}\PY{p}{(}\PY{l+m+mi}{2}\PY{p}{,}\PY{l+m+mi}{2}\PY{p}{,}\PY{l+m+mi}{1}\PY{p}{)}
\PY{n}{means}\PY{p}{[}\PY{l+s+s1}{\PYZsq{}}\PY{l+s+s1}{fn1\PYZus{}m}\PY{l+s+s1}{\PYZsq{}}\PY{p}{]}\PY{o}{.}\PY{n}{plot}\PY{p}{(}\PY{p}{)}
\PY{n}{plt}\PY{o}{.}\PY{n}{title}\PY{p}{(}\PY{l+s+s1}{\PYZsq{}}\PY{l+s+s1}{Primeira frequência}\PY{l+s+s1}{\PYZsq{}}\PY{p}{)}
\PY{n}{plt}\PY{o}{.}\PY{n}{ylabel}\PY{p}{(}\PY{l+s+s1}{\PYZsq{}}\PY{l+s+s1}{\PYZdl{}}\PY{l+s+s1}{\PYZbs{}}\PY{l+s+s1}{mu\PYZus{}}\PY{l+s+si}{\PYZob{}f1\PYZcb{}}\PY{l+s+s1}{\PYZdl{}}\PY{l+s+s1}{\PYZsq{}}\PY{p}{)}
\PY{n}{plt}\PY{o}{.}\PY{n}{xscale}\PY{p}{(}\PY{l+s+s1}{\PYZsq{}}\PY{l+s+s1}{log}\PY{l+s+s1}{\PYZsq{}}\PY{p}{)}
\PY{n}{plt}\PY{o}{.}\PY{n}{xlim}\PY{p}{(}\PY{n}{minp}\PY{p}{,}\PY{n}{N}\PY{p}{)}
\PY{n}{plt}\PY{o}{.}\PY{n}{grid}\PY{p}{(}\PY{k+kc}{True}\PY{p}{)}

\PY{n}{pl}  \PY{o}{=} \PY{n}{plt}\PY{o}{.}\PY{n}{subplot}\PY{p}{(}\PY{l+m+mi}{2}\PY{p}{,}\PY{l+m+mi}{2}\PY{p}{,}\PY{l+m+mi}{2}\PY{p}{)}
\PY{n}{means}\PY{p}{[}\PY{l+s+s1}{\PYZsq{}}\PY{l+s+s1}{fn2\PYZus{}m}\PY{l+s+s1}{\PYZsq{}}\PY{p}{]}\PY{o}{.}\PY{n}{plot}\PY{p}{(}\PY{p}{)}
\PY{n}{plt}\PY{o}{.}\PY{n}{title}\PY{p}{(}\PY{l+s+s1}{\PYZsq{}}\PY{l+s+s1}{Segunda frequência}\PY{l+s+s1}{\PYZsq{}}\PY{p}{)}
\PY{n}{plt}\PY{o}{.}\PY{n}{ylabel}\PY{p}{(}\PY{l+s+s1}{\PYZsq{}}\PY{l+s+s1}{\PYZdl{}}\PY{l+s+s1}{\PYZbs{}}\PY{l+s+s1}{mu\PYZus{}}\PY{l+s+si}{\PYZob{}f2\PYZcb{}}\PY{l+s+s1}{\PYZdl{}}\PY{l+s+s1}{\PYZsq{}}\PY{p}{)}
\PY{n}{plt}\PY{o}{.}\PY{n}{xscale}\PY{p}{(}\PY{l+s+s1}{\PYZsq{}}\PY{l+s+s1}{log}\PY{l+s+s1}{\PYZsq{}}\PY{p}{)}
\PY{n}{plt}\PY{o}{.}\PY{n}{xlim}\PY{p}{(}\PY{n}{minp}\PY{p}{,}\PY{n}{N}\PY{p}{)}
\PY{n}{plt}\PY{o}{.}\PY{n}{grid}\PY{p}{(}\PY{k+kc}{True}\PY{p}{)}

\PY{n}{pl}  \PY{o}{=} \PY{n}{plt}\PY{o}{.}\PY{n}{subplot}\PY{p}{(}\PY{l+m+mi}{2}\PY{p}{,}\PY{l+m+mi}{2}\PY{p}{,}\PY{l+m+mi}{3}\PY{p}{)}
\PY{n}{stds}\PY{p}{[}\PY{l+s+s1}{\PYZsq{}}\PY{l+s+s1}{fn1\PYZus{}m}\PY{l+s+s1}{\PYZsq{}}\PY{p}{]}\PY{o}{.}\PY{n}{plot}\PY{p}{(}\PY{p}{)}
\PY{n}{plt}\PY{o}{.}\PY{n}{xlabel}\PY{p}{(}\PY{l+s+s1}{\PYZsq{}}\PY{l+s+s1}{Simulações}\PY{l+s+s1}{\PYZsq{}}\PY{p}{)}
\PY{n}{plt}\PY{o}{.}\PY{n}{ylabel}\PY{p}{(}\PY{l+s+s1}{\PYZsq{}}\PY{l+s+s1}{\PYZdl{}}\PY{l+s+s1}{\PYZbs{}}\PY{l+s+s1}{sigma\PYZus{}}\PY{l+s+si}{\PYZob{}f1\PYZcb{}}\PY{l+s+s1}{\PYZdl{}}\PY{l+s+s1}{\PYZsq{}}\PY{p}{)}
\PY{n}{plt}\PY{o}{.}\PY{n}{xscale}\PY{p}{(}\PY{l+s+s1}{\PYZsq{}}\PY{l+s+s1}{log}\PY{l+s+s1}{\PYZsq{}}\PY{p}{)}
\PY{n}{plt}\PY{o}{.}\PY{n}{xlim}\PY{p}{(}\PY{n}{minp}\PY{p}{,}\PY{n}{N}\PY{p}{)}
\PY{n}{plt}\PY{o}{.}\PY{n}{grid}\PY{p}{(}\PY{k+kc}{True}\PY{p}{)}

\PY{n}{pl}  \PY{o}{=} \PY{n}{plt}\PY{o}{.}\PY{n}{subplot}\PY{p}{(}\PY{l+m+mi}{2}\PY{p}{,}\PY{l+m+mi}{2}\PY{p}{,}\PY{l+m+mi}{4}\PY{p}{)}
\PY{n}{stds}\PY{p}{[}\PY{l+s+s1}{\PYZsq{}}\PY{l+s+s1}{fn2\PYZus{}m}\PY{l+s+s1}{\PYZsq{}}\PY{p}{]}\PY{o}{.}\PY{n}{plot}\PY{p}{(}\PY{p}{)}
\PY{n}{plt}\PY{o}{.}\PY{n}{xlabel}\PY{p}{(}\PY{l+s+s1}{\PYZsq{}}\PY{l+s+s1}{Simulações}\PY{l+s+s1}{\PYZsq{}}\PY{p}{)}
\PY{n}{plt}\PY{o}{.}\PY{n}{ylabel}\PY{p}{(}\PY{l+s+s1}{\PYZsq{}}\PY{l+s+s1}{\PYZdl{}}\PY{l+s+s1}{\PYZbs{}}\PY{l+s+s1}{sigma\PYZus{}}\PY{l+s+si}{\PYZob{}f2\PYZcb{}}\PY{l+s+s1}{\PYZdl{}}\PY{l+s+s1}{\PYZsq{}}\PY{p}{)}
\PY{n}{plt}\PY{o}{.}\PY{n}{xscale}\PY{p}{(}\PY{l+s+s1}{\PYZsq{}}\PY{l+s+s1}{log}\PY{l+s+s1}{\PYZsq{}}\PY{p}{)}
\PY{n}{plt}\PY{o}{.}\PY{n}{xlim}\PY{p}{(}\PY{n}{minp}\PY{p}{,}\PY{n}{N}\PY{p}{)}
\PY{n}{plt}\PY{o}{.}\PY{n}{grid}\PY{p}{(}\PY{k+kc}{True}\PY{p}{)}
\end{Verbatim}
\end{tcolorbox}

    \begin{center}
    \adjustimage{max size={0.9\linewidth}{0.9\paperheight}}{trab3a_files/trab3a_18_0.pdf}
    \end{center}
    { \hspace*{\fill} \\}
    
    Conforme a figura, as médias e desvios padrões das frequências de
vibração do modelo reduzido estabilizam com cerca de \(2\times 10^5\)
simulações, sendo esse valor observado para diferentes sementes
geradoras de números aleatórios. Dada a convergência do processo os
valores apresentados anteriormente, na descrição dos resultados, são
adotados, dessa forma:

    \begin{tcolorbox}[breakable, size=fbox, boxrule=1pt, pad at break*=1mm,colback=cellbackground, colframe=cellborder]
\prompt{In}{incolor}{10}{\boxspacing}
\begin{Verbatim}[commandchars=\\\{\}]
\PY{n}{mean\PYZus{}f1} \PY{o}{=} \PY{n}{desc\PYZus{}geral}\PY{o}{.}\PY{n}{loc}\PY{p}{[}\PY{l+s+s1}{\PYZsq{}}\PY{l+s+s1}{mean}\PY{l+s+s1}{\PYZsq{}}\PY{p}{]}\PY{p}{[}\PY{l+s+s1}{\PYZsq{}}\PY{l+s+s1}{fn1\PYZus{}m}\PY{l+s+s1}{\PYZsq{}}\PY{p}{]}
\PY{n}{mean\PYZus{}f2} \PY{o}{=} \PY{n}{desc\PYZus{}geral}\PY{o}{.}\PY{n}{loc}\PY{p}{[}\PY{l+s+s1}{\PYZsq{}}\PY{l+s+s1}{mean}\PY{l+s+s1}{\PYZsq{}}\PY{p}{]}\PY{p}{[}\PY{l+s+s1}{\PYZsq{}}\PY{l+s+s1}{fn2\PYZus{}m}\PY{l+s+s1}{\PYZsq{}}\PY{p}{]}
\PY{n}{std\PYZus{}f1}  \PY{o}{=} \PY{n}{desc\PYZus{}geral}\PY{o}{.}\PY{n}{loc}\PY{p}{[}\PY{l+s+s1}{\PYZsq{}}\PY{l+s+s1}{std}\PY{l+s+s1}{\PYZsq{}}\PY{p}{]}\PY{p}{[}\PY{l+s+s1}{\PYZsq{}}\PY{l+s+s1}{fn1\PYZus{}m}\PY{l+s+s1}{\PYZsq{}}\PY{p}{]}
\PY{n}{std\PYZus{}f2}  \PY{o}{=} \PY{n}{desc\PYZus{}geral}\PY{o}{.}\PY{n}{loc}\PY{p}{[}\PY{l+s+s1}{\PYZsq{}}\PY{l+s+s1}{std}\PY{l+s+s1}{\PYZsq{}}\PY{p}{]}\PY{p}{[}\PY{l+s+s1}{\PYZsq{}}\PY{l+s+s1}{fn2\PYZus{}m}\PY{l+s+s1}{\PYZsq{}}\PY{p}{]}


\PY{n+nb}{print}\PY{p}{(}\PY{l+s+s1}{\PYZsq{}\PYZsq{}\PYZsq{}}\PY{l+s+s1}{Considerando as incertezas existentes:}
\PY{l+s+s1}{ \PYZhy{} A primeira frequência de vibração do modelo reduzido é }\PY{l+s+si}{\PYZob{}f1:.2f\PYZcb{}}\PY{l+s+s1}{ Hz com um desvio padrão de }\PY{l+s+si}{\PYZob{}f1s:.2f\PYZcb{}}\PY{l+s+s1}{ Hz}
\PY{l+s+s1}{ \PYZhy{} A segunda frequência de vibração do modelo reduzido é }\PY{l+s+si}{\PYZob{}f2:.2f\PYZcb{}}\PY{l+s+s1}{ Hz com um desvio padrão de }\PY{l+s+si}{\PYZob{}f2s:.2f\PYZcb{}}\PY{l+s+s1}{ Hz}
\PY{l+s+s1}{ }\PY{l+s+s1}{\PYZsq{}\PYZsq{}\PYZsq{}}\PY{o}{.}\PY{n}{format}\PY{p}{(}\PY{n}{f1}\PY{o}{=}\PY{n}{mean\PYZus{}f1}\PY{p}{,} \PY{n}{f2}\PY{o}{=}\PY{n}{mean\PYZus{}f2}\PY{p}{,} \PY{n}{f1s}\PY{o}{=}\PY{n}{std\PYZus{}f1}\PY{p}{,} \PY{n}{f2s}\PY{o}{=}\PY{n}{std\PYZus{}f2}\PY{p}{)}\PY{p}{)}
\end{Verbatim}
\end{tcolorbox}

    \begin{Verbatim}[commandchars=\\\{\}]
Considerando as incertezas existentes:
 - A primeira frequência de vibração do modelo reduzido é 18.67 Hz com um desvio
padrão de 1.21 Hz
 - A segunda frequência de vibração do modelo reduzido é 116.16 Hz com um desvio
padrão de 7.52 Hz

    \end{Verbatim}

    Como a média e o desvio padrão dos resultados são conhecidos os
histogramas dos resultados podem ser comparados às distribuições normais
formadas por tais parâmetros, avaliando diferentes números de
simulações:

    \begin{tcolorbox}[breakable, size=fbox, boxrule=1pt, pad at break*=1mm,colback=cellbackground, colframe=cellborder]
\prompt{In}{incolor}{11}{\boxspacing}
\begin{Verbatim}[commandchars=\\\{\}]
\PY{n}{rv\PYZus{}f1} \PY{o}{=} \PY{n}{st}\PY{o}{.}\PY{n}{norm}\PY{p}{(}\PY{n}{mean\PYZus{}f1}\PY{p}{,} \PY{n}{std\PYZus{}f1}\PY{p}{)}
\PY{n}{rv\PYZus{}f2} \PY{o}{=} \PY{n}{st}\PY{o}{.}\PY{n}{norm}\PY{p}{(}\PY{n}{mean\PYZus{}f2}\PY{p}{,} \PY{n}{std\PYZus{}f2}\PY{p}{)}

\PY{n}{k} \PY{o}{=} \PY{l+m+mi}{4}
\PY{n}{x1} \PY{o}{=} \PY{n}{np}\PY{o}{.}\PY{n}{linspace}\PY{p}{(}\PY{n}{mean\PYZus{}f1} \PY{o}{\PYZhy{}} \PY{n}{k}\PY{o}{*}\PY{n}{std\PYZus{}f1}\PY{p}{,} \PY{n}{mean\PYZus{}f1} \PY{o}{+} \PY{n}{k}\PY{o}{*}\PY{n}{std\PYZus{}f1}\PY{p}{,} \PY{l+m+mi}{500}\PY{p}{)}
\PY{n}{x2} \PY{o}{=} \PY{n}{np}\PY{o}{.}\PY{n}{linspace}\PY{p}{(}\PY{n}{mean\PYZus{}f2} \PY{o}{\PYZhy{}} \PY{n}{k}\PY{o}{*}\PY{n}{std\PYZus{}f2}\PY{p}{,} \PY{n}{mean\PYZus{}f2} \PY{o}{+} \PY{n}{k}\PY{o}{*}\PY{n}{std\PYZus{}f2}\PY{p}{,} \PY{l+m+mi}{500}\PY{p}{)}

\PY{n}{Nhs} \PY{o}{=} \PY{p}{[}\PY{l+m+mi}{100}\PY{p}{,} \PY{l+m+mi}{1000}\PY{p}{,} \PY{l+m+mi}{10000}\PY{p}{,} \PY{n+nb}{int}\PY{p}{(}\PY{l+m+mf}{1e6}\PY{p}{)}\PY{p}{]}

\PY{n}{figHists} \PY{o}{=} \PY{n}{plt}\PY{o}{.}\PY{n}{figure}\PY{p}{(}\PY{l+m+mi}{2}\PY{p}{,} \PY{n}{figsize}\PY{o}{=}\PY{p}{(}\PY{l+m+mi}{8}\PY{p}{,}\PY{l+m+mi}{8}\PY{p}{)}\PY{p}{)}
\PY{n}{figHists}\PY{o}{.}\PY{n}{suptitle}\PY{p}{(}\PY{l+s+s1}{\PYZsq{}}\PY{l+s+s1}{Primeira Frequência Natural}\PY{l+s+s1}{\PYZsq{}}\PY{p}{)}
\PY{k}{for} \PY{n}{i}\PY{p}{,} \PY{n}{Nh} \PY{o+ow}{in} \PY{n+nb}{enumerate}\PY{p}{(}\PY{n}{Nhs}\PY{p}{)}\PY{p}{:}
    \PY{n}{pl} \PY{o}{=} \PY{n}{plt}\PY{o}{.}\PY{n}{subplot}\PY{p}{(}\PY{n+nb}{len}\PY{p}{(}\PY{n}{Nhs}\PY{p}{)}\PY{p}{,} \PY{l+m+mi}{2}\PY{p}{,} \PY{l+m+mi}{2}\PY{o}{*}\PY{p}{(}\PY{n}{i}\PY{o}{+}\PY{l+m+mi}{1}\PY{p}{)}\PY{o}{\PYZhy{}}\PY{l+m+mi}{1}\PY{p}{)}
    \PY{k}{if} \PY{n}{i}\PY{o}{==}\PY{l+m+mi}{0}\PY{p}{:} \PY{n}{plt}\PY{o}{.}\PY{n}{title}\PY{p}{(}\PY{l+s+s1}{\PYZsq{}}\PY{l+s+s1}{Densidade de Probabilidade}\PY{l+s+s1}{\PYZsq{}}\PY{p}{,} \PY{n}{size}\PY{o}{=}\PY{l+m+mi}{10}\PY{p}{)}
    \PY{n}{plt}\PY{o}{.}\PY{n}{plot}\PY{p}{(}\PY{n}{x1}\PY{p}{,} \PY{n}{rv\PYZus{}f1}\PY{o}{.}\PY{n}{pdf}\PY{p}{(}\PY{n}{x1}\PY{p}{)}\PY{p}{)}
    \PY{n}{dados}\PY{p}{[}\PY{l+s+s1}{\PYZsq{}}\PY{l+s+s1}{fn1\PYZus{}m}\PY{l+s+s1}{\PYZsq{}}\PY{p}{]}\PY{p}{[}\PY{l+m+mi}{0}\PY{p}{:}\PY{n}{Nh}\PY{p}{]}\PY{o}{.}\PY{n}{plot}\PY{p}{(}\PY{n}{kind}\PY{o}{=}\PY{l+s+s1}{\PYZsq{}}\PY{l+s+s1}{hist}\PY{l+s+s1}{\PYZsq{}}\PY{p}{,} \PY{n}{density}\PY{o}{=}\PY{k+kc}{True}\PY{p}{,} \PY{n}{bins}\PY{o}{=}\PY{l+m+mi}{20}\PY{p}{,} \PY{n}{label}\PY{o}{=}\PY{l+s+sa}{f}\PY{l+s+s1}{\PYZsq{}}\PY{l+s+s1}{N=}\PY{l+s+si}{\PYZob{}}\PY{n}{Nh}\PY{l+s+si}{:}\PY{l+s+s1}{.1E}\PY{l+s+si}{\PYZcb{}}\PY{l+s+s1}{\PYZsq{}}\PY{p}{)}
    \PY{n}{plt}\PY{o}{.}\PY{n}{ylabel}\PY{p}{(}\PY{k+kc}{None}\PY{p}{)}
    \PY{k}{if} \PY{n}{i}\PY{o}{+}\PY{l+m+mi}{1}\PY{o}{==}\PY{n+nb}{len}\PY{p}{(}\PY{n}{Nhs}\PY{p}{)}\PY{p}{:} \PY{n}{plt}\PY{o}{.}\PY{n}{xlabel}\PY{p}{(}\PY{l+s+s1}{\PYZsq{}}\PY{l+s+s1}{Frquência em Hz}\PY{l+s+s1}{\PYZsq{}}\PY{p}{)}
    \PY{n}{plt}\PY{o}{.}\PY{n}{legend}\PY{p}{(}\PY{n}{loc}\PY{o}{=}\PY{l+s+s1}{\PYZsq{}}\PY{l+s+s1}{upper left}\PY{l+s+s1}{\PYZsq{}}\PY{p}{,} \PY{n}{fontsize}\PY{o}{=}\PY{l+m+mi}{8}\PY{p}{)}

    \PY{n}{pl} \PY{o}{=} \PY{n}{plt}\PY{o}{.}\PY{n}{subplot}\PY{p}{(}\PY{n+nb}{len}\PY{p}{(}\PY{n}{Nhs}\PY{p}{)}\PY{p}{,} \PY{l+m+mi}{2}\PY{p}{,} \PY{l+m+mi}{2}\PY{o}{*}\PY{p}{(}\PY{n}{i}\PY{o}{+}\PY{l+m+mi}{1}\PY{p}{)}\PY{p}{)}
    \PY{k}{if} \PY{n}{i}\PY{o}{==}\PY{l+m+mi}{0}\PY{p}{:} \PY{n}{plt}\PY{o}{.}\PY{n}{title}\PY{p}{(}\PY{l+s+s1}{\PYZsq{}}\PY{l+s+s1}{Densidade de Probabilidade Acumulada}\PY{l+s+s1}{\PYZsq{}}\PY{p}{,} \PY{n}{size}\PY{o}{=}\PY{l+m+mi}{10}\PY{p}{)}
    \PY{n}{plt}\PY{o}{.}\PY{n}{plot}\PY{p}{(}\PY{n}{x1}\PY{p}{,} \PY{n}{rv\PYZus{}f1}\PY{o}{.}\PY{n}{cdf}\PY{p}{(}\PY{n}{x1}\PY{p}{)}\PY{p}{)}
    \PY{n}{dados}\PY{p}{[}\PY{l+s+s1}{\PYZsq{}}\PY{l+s+s1}{fn1\PYZus{}m}\PY{l+s+s1}{\PYZsq{}}\PY{p}{]}\PY{p}{[}\PY{l+m+mi}{0}\PY{p}{:}\PY{n}{Nh}\PY{p}{]}\PY{o}{.}\PY{n}{plot}\PY{p}{(}\PY{n}{kind}\PY{o}{=}\PY{l+s+s1}{\PYZsq{}}\PY{l+s+s1}{hist}\PY{l+s+s1}{\PYZsq{}}\PY{p}{,} \PY{n}{density}\PY{o}{=}\PY{k+kc}{True}\PY{p}{,} \PY{n}{bins}\PY{o}{=}\PY{l+m+mi}{20}\PY{p}{,} \PY{n}{cumulative}\PY{o}{=}\PY{k+kc}{True}\PY{p}{)}
    \PY{n}{plt}\PY{o}{.}\PY{n}{ylabel}\PY{p}{(}\PY{k+kc}{None}\PY{p}{)}
    \PY{k}{if} \PY{n}{i}\PY{o}{+}\PY{l+m+mi}{1}\PY{o}{==}\PY{n+nb}{len}\PY{p}{(}\PY{n}{Nhs}\PY{p}{)}\PY{p}{:} \PY{n}{plt}\PY{o}{.}\PY{n}{xlabel}\PY{p}{(}\PY{l+s+s1}{\PYZsq{}}\PY{l+s+s1}{Frquência em Hz}\PY{l+s+s1}{\PYZsq{}}\PY{p}{)}
\end{Verbatim}
\end{tcolorbox}

    \begin{center}
    \adjustimage{max size={0.9\linewidth}{0.9\paperheight}}{trab3a_files/trab3a_22_0.pdf}
    \end{center}
    { \hspace*{\fill} \\}
    
    \begin{tcolorbox}[breakable, size=fbox, boxrule=1pt, pad at break*=1mm,colback=cellbackground, colframe=cellborder]
\prompt{In}{incolor}{12}{\boxspacing}
\begin{Verbatim}[commandchars=\\\{\}]
\PY{n}{figHists} \PY{o}{=} \PY{n}{plt}\PY{o}{.}\PY{n}{figure}\PY{p}{(}\PY{l+m+mi}{3}\PY{p}{,} \PY{n}{figsize}\PY{o}{=}\PY{p}{(}\PY{l+m+mi}{8}\PY{p}{,}\PY{l+m+mi}{8}\PY{p}{)}\PY{p}{)}
\PY{n}{figHists}\PY{o}{.}\PY{n}{suptitle}\PY{p}{(}\PY{l+s+s1}{\PYZsq{}}\PY{l+s+s1}{Segunda Frequência Natural}\PY{l+s+s1}{\PYZsq{}}\PY{p}{)}
\PY{k}{for} \PY{n}{i}\PY{p}{,} \PY{n}{Nh} \PY{o+ow}{in} \PY{n+nb}{enumerate}\PY{p}{(}\PY{n}{Nhs}\PY{p}{)}\PY{p}{:}
    \PY{n}{pl} \PY{o}{=} \PY{n}{plt}\PY{o}{.}\PY{n}{subplot}\PY{p}{(}\PY{n+nb}{len}\PY{p}{(}\PY{n}{Nhs}\PY{p}{)}\PY{p}{,} \PY{l+m+mi}{2}\PY{p}{,} \PY{l+m+mi}{2}\PY{o}{*}\PY{p}{(}\PY{n}{i}\PY{o}{+}\PY{l+m+mi}{1}\PY{p}{)}\PY{o}{\PYZhy{}}\PY{l+m+mi}{1}\PY{p}{)}
    \PY{k}{if} \PY{n}{i}\PY{o}{==}\PY{l+m+mi}{0}\PY{p}{:} \PY{n}{plt}\PY{o}{.}\PY{n}{title}\PY{p}{(}\PY{l+s+s1}{\PYZsq{}}\PY{l+s+s1}{Densidade de Probabilidade}\PY{l+s+s1}{\PYZsq{}}\PY{p}{,} \PY{n}{size}\PY{o}{=}\PY{l+m+mi}{10}\PY{p}{)}
    \PY{n}{plt}\PY{o}{.}\PY{n}{plot}\PY{p}{(}\PY{n}{x2}\PY{p}{,} \PY{n}{rv\PYZus{}f2}\PY{o}{.}\PY{n}{pdf}\PY{p}{(}\PY{n}{x2}\PY{p}{)}\PY{p}{)}
    \PY{n}{dados}\PY{p}{[}\PY{l+s+s1}{\PYZsq{}}\PY{l+s+s1}{fn2\PYZus{}m}\PY{l+s+s1}{\PYZsq{}}\PY{p}{]}\PY{p}{[}\PY{l+m+mi}{0}\PY{p}{:}\PY{n}{Nh}\PY{p}{]}\PY{o}{.}\PY{n}{plot}\PY{p}{(}\PY{n}{kind}\PY{o}{=}\PY{l+s+s1}{\PYZsq{}}\PY{l+s+s1}{hist}\PY{l+s+s1}{\PYZsq{}}\PY{p}{,} \PY{n}{density}\PY{o}{=}\PY{k+kc}{True}\PY{p}{,} \PY{n}{bins}\PY{o}{=}\PY{l+m+mi}{20}\PY{p}{,} \PY{n}{label}\PY{o}{=}\PY{l+s+sa}{f}\PY{l+s+s1}{\PYZsq{}}\PY{l+s+s1}{N=}\PY{l+s+si}{\PYZob{}}\PY{n}{Nh}\PY{l+s+si}{:}\PY{l+s+s1}{.1E}\PY{l+s+si}{\PYZcb{}}\PY{l+s+s1}{\PYZsq{}}\PY{p}{)}
    \PY{n}{plt}\PY{o}{.}\PY{n}{ylabel}\PY{p}{(}\PY{k+kc}{None}\PY{p}{)}
    \PY{k}{if} \PY{n}{i}\PY{o}{+}\PY{l+m+mi}{1}\PY{o}{==}\PY{n+nb}{len}\PY{p}{(}\PY{n}{Nhs}\PY{p}{)}\PY{p}{:} \PY{n}{plt}\PY{o}{.}\PY{n}{xlabel}\PY{p}{(}\PY{l+s+s1}{\PYZsq{}}\PY{l+s+s1}{Frquência em Hz}\PY{l+s+s1}{\PYZsq{}}\PY{p}{)}
    \PY{n}{plt}\PY{o}{.}\PY{n}{legend}\PY{p}{(}\PY{n}{loc}\PY{o}{=}\PY{l+s+s1}{\PYZsq{}}\PY{l+s+s1}{upper left}\PY{l+s+s1}{\PYZsq{}}\PY{p}{,} \PY{n}{fontsize}\PY{o}{=}\PY{l+m+mi}{8}\PY{p}{)}

    \PY{n}{pl} \PY{o}{=} \PY{n}{plt}\PY{o}{.}\PY{n}{subplot}\PY{p}{(}\PY{n+nb}{len}\PY{p}{(}\PY{n}{Nhs}\PY{p}{)}\PY{p}{,} \PY{l+m+mi}{2}\PY{p}{,} \PY{l+m+mi}{2}\PY{o}{*}\PY{p}{(}\PY{n}{i}\PY{o}{+}\PY{l+m+mi}{1}\PY{p}{)}\PY{p}{)}
    \PY{k}{if} \PY{n}{i}\PY{o}{==}\PY{l+m+mi}{0}\PY{p}{:} \PY{n}{plt}\PY{o}{.}\PY{n}{title}\PY{p}{(}\PY{l+s+s1}{\PYZsq{}}\PY{l+s+s1}{Densidade de Probabilidade Acumulada}\PY{l+s+s1}{\PYZsq{}}\PY{p}{,} \PY{n}{size}\PY{o}{=}\PY{l+m+mi}{10}\PY{p}{)}
    \PY{n}{plt}\PY{o}{.}\PY{n}{plot}\PY{p}{(}\PY{n}{x2}\PY{p}{,} \PY{n}{rv\PYZus{}f2}\PY{o}{.}\PY{n}{cdf}\PY{p}{(}\PY{n}{x2}\PY{p}{)}\PY{p}{)}
    \PY{n}{dados}\PY{p}{[}\PY{l+s+s1}{\PYZsq{}}\PY{l+s+s1}{fn2\PYZus{}m}\PY{l+s+s1}{\PYZsq{}}\PY{p}{]}\PY{p}{[}\PY{l+m+mi}{0}\PY{p}{:}\PY{n}{Nh}\PY{p}{]}\PY{o}{.}\PY{n}{plot}\PY{p}{(}\PY{n}{kind}\PY{o}{=}\PY{l+s+s1}{\PYZsq{}}\PY{l+s+s1}{hist}\PY{l+s+s1}{\PYZsq{}}\PY{p}{,} \PY{n}{density}\PY{o}{=}\PY{k+kc}{True}\PY{p}{,} \PY{n}{bins}\PY{o}{=}\PY{l+m+mi}{20}\PY{p}{,} \PY{n}{cumulative}\PY{o}{=}\PY{k+kc}{True}\PY{p}{)}
    \PY{n}{plt}\PY{o}{.}\PY{n}{ylabel}\PY{p}{(}\PY{k+kc}{None}\PY{p}{)}
    \PY{k}{if} \PY{n}{i}\PY{o}{+}\PY{l+m+mi}{1}\PY{o}{==}\PY{n+nb}{len}\PY{p}{(}\PY{n}{Nhs}\PY{p}{)}\PY{p}{:} \PY{n}{plt}\PY{o}{.}\PY{n}{xlabel}\PY{p}{(}\PY{l+s+s1}{\PYZsq{}}\PY{l+s+s1}{Frquência em Hz}\PY{l+s+s1}{\PYZsq{}}\PY{p}{)}
\end{Verbatim}
\end{tcolorbox}

    \begin{center}
    \adjustimage{max size={0.9\linewidth}{0.9\paperheight}}{trab3a_files/trab3a_23_0.pdf}
    \end{center}
    { \hspace*{\fill} \\}
    
    Conforme esperado, o aumento do número de simulações consideradas na
construção dos histogramas melhorou a suavidade desses. Em relação à
distribuição dos resultados, mesmo com uma função não-linear é observada
boa aderência desses à distribuição normal.

Sendo o erro propagado uma variável aleatória de distribuição normal,
podem ser determinados intervalos de confiança das frequências do modelo
reduzido, considerando erros bilaterais, por:

\begin{equation}
\mu_{fi} - k \sigma_{fi} \le f_i \le \mu_{fi} + k \sigma_{fi}
\end{equation}

onde \(k\) é obtido através do inverso da distribuição de probabilidade
acumulada para um dado erro \(\varepsilon\) tal que:

\begin{equation}
k = -\Phi^{-1}(\varepsilon/2)
\end{equation}

Supondo erros bilaterais aceitáveis de 2.5\%, 5\%, 10\% e 50\% os
limites obtidos são:

    \begin{tcolorbox}[breakable, size=fbox, boxrule=1pt, pad at break*=1mm,colback=cellbackground, colframe=cellborder]
\prompt{In}{incolor}{13}{\boxspacing}
\begin{Verbatim}[commandchars=\\\{\}]
\PY{k}{def} \PY{n+nf}{IC}\PY{p}{(}\PY{n}{mean}\PY{p}{,} \PY{n}{std}\PY{p}{,} \PY{n}{err}\PY{p}{)}\PY{p}{:}
    \PY{n}{k} \PY{o}{=} \PY{o}{\PYZhy{}}\PY{n}{st}\PY{o}{.}\PY{n}{norm}\PY{o}{.}\PY{n}{ppf}\PY{p}{(}\PY{n}{err}\PY{o}{/}\PY{l+m+mi}{2}\PY{p}{)} \PY{c+c1}{\PYZsh{} Bilateral}
    \PY{k}{return} \PY{n}{mean}\PY{o}{\PYZhy{}}\PY{n}{k}\PY{o}{*}\PY{n}{std}\PY{p}{,} \PY{n}{mean}\PY{o}{+}\PY{n}{k}\PY{o}{*}\PY{n}{std}\PY{p}{,} \PY{n}{k}

\PY{k}{def} \PY{n+nf}{printIC}\PY{p}{(}\PY{n}{err}\PY{p}{)}\PY{p}{:}
    \PY{n}{f1min}\PY{p}{,} \PY{n}{f1max}\PY{p}{,} \PY{n}{k} \PY{o}{=} \PY{n}{IC}\PY{p}{(}\PY{n}{mean\PYZus{}f1}\PY{p}{,} \PY{n}{std\PYZus{}f1}\PY{p}{,} \PY{n}{err}\PY{p}{)}
    \PY{n}{f2min}\PY{p}{,} \PY{n}{f2max}\PY{p}{,} \PY{n}{k} \PY{o}{=} \PY{n}{IC}\PY{p}{(}\PY{n}{mean\PYZus{}f2}\PY{p}{,} \PY{n}{std\PYZus{}f2}\PY{p}{,} \PY{n}{err}\PY{p}{)}
    \PY{n+nb}{print}\PY{p}{(}\PY{l+s+sa}{f}\PY{l+s+s1}{\PYZsq{}\PYZsq{}\PYZsq{}}\PY{l+s+s1}{Com uma confiança de }\PY{l+s+si}{\PYZob{}}\PY{l+m+mi}{1}\PY{o}{\PYZhy{}}\PY{n}{err}\PY{l+s+si}{:}\PY{l+s+s1}{.2\PYZpc{}}\PY{l+s+si}{\PYZcb{}}\PY{l+s+s1}{: (k=}\PY{l+s+si}{\PYZob{}}\PY{n}{k}\PY{l+s+si}{:}\PY{l+s+s1}{.3f}\PY{l+s+si}{\PYZcb{}}\PY{l+s+s1}{)}
\PY{l+s+s1}{    * }\PY{l+s+si}{\PYZob{}}\PY{n}{f1min}\PY{l+s+si}{:}\PY{l+s+s1}{.2f}\PY{l+s+si}{\PYZcb{}}\PY{l+s+s1}{ Hz \PYZlt{}= f1 \PYZlt{}= }\PY{l+s+si}{\PYZob{}}\PY{n}{f1max}\PY{l+s+si}{:}\PY{l+s+s1}{.2f}\PY{l+s+si}{\PYZcb{}}\PY{l+s+s1}{ Hz}
\PY{l+s+s1}{    * }\PY{l+s+si}{\PYZob{}}\PY{n}{f2min}\PY{l+s+si}{:}\PY{l+s+s1}{.2f}\PY{l+s+si}{\PYZcb{}}\PY{l+s+s1}{ Hz \PYZlt{}= f2 \PYZlt{}= }\PY{l+s+si}{\PYZob{}}\PY{n}{f2max}\PY{l+s+si}{:}\PY{l+s+s1}{.2f}\PY{l+s+si}{\PYZcb{}}\PY{l+s+s1}{ Hz}
\PY{l+s+s1}{    }\PY{l+s+s1}{\PYZsq{}\PYZsq{}\PYZsq{}}\PY{p}{)}
    
\PY{n}{printIC}\PY{p}{(}\PY{l+m+mf}{0.025}\PY{p}{)}
\PY{n}{printIC}\PY{p}{(}\PY{l+m+mf}{0.050}\PY{p}{)}
\PY{n}{printIC}\PY{p}{(}\PY{l+m+mf}{0.100}\PY{p}{)}
\PY{n}{printIC}\PY{p}{(}\PY{l+m+mf}{0.500}\PY{p}{)}
\end{Verbatim}
\end{tcolorbox}

    \begin{Verbatim}[commandchars=\\\{\}]
Com uma confiança de 97.50\%: (k=2.241)
    * 15.96 Hz <= f1 <= 21.38 Hz
    * 99.30 Hz <= f2 <= 133.03 Hz

Com uma confiança de 95.00\%: (k=1.960)
    * 16.30 Hz <= f1 <= 21.03 Hz
    * 101.41 Hz <= f2 <= 130.91 Hz

Com uma confiança de 90.00\%: (k=1.645)
    * 16.68 Hz <= f1 <= 20.65 Hz
    * 103.78 Hz <= f2 <= 128.54 Hz

Com uma confiança de 50.00\%: (k=0.674)
    * 17.85 Hz <= f1 <= 19.48 Hz
    * 111.09 Hz <= f2 <= 121.24 Hz

    \end{Verbatim}

    Evidentemente, a aceitação de um erro maior reduz o tamanho do intervalo
de frequências possíveis. Dos resultados apresentados acima, adotando a
confiança de 95\%, ou probabilidade do valor estar fora do intervalo de
5\%, as frequências do modelo estão nos intervalos:

\[ 16.30 Hz \le f_1 \le 21.03 Hz \]

\[ 101.41 Hz \le f_2 \le 130.91 Hz \]

Os intervalos podem ainda ser ilustrados sobre a curva de densidade de
probabilidade das distribuições de frequências:

    \begin{tcolorbox}[breakable, size=fbox, boxrule=1pt, pad at break*=1mm,colback=cellbackground, colframe=cellborder]
\prompt{In}{incolor}{14}{\boxspacing}
\begin{Verbatim}[commandchars=\\\{\}]
\PY{k}{def} \PY{n+nf}{plotIC}\PY{p}{(}\PY{n}{mean}\PY{p}{,} \PY{n}{std}\PY{p}{,} \PY{n}{err}\PY{p}{)}\PY{p}{:}
    \PY{n}{limmin} \PY{o}{=} \PY{n}{mean} \PY{o}{\PYZhy{}} \PY{n}{k}\PY{o}{*}\PY{n}{std}
    \PY{n}{limmax} \PY{o}{=} \PY{n}{mean} \PY{o}{+} \PY{n}{k}\PY{o}{*}\PY{n}{std}
    
    \PY{n}{ea}     \PY{o}{=} \PY{l+m+mi}{0}
    \PY{n}{colors} \PY{o}{=} \PY{p}{[}\PY{l+s+s1}{\PYZsq{}}\PY{l+s+s1}{g}\PY{l+s+s1}{\PYZsq{}}\PY{p}{,} \PY{l+s+s1}{\PYZsq{}}\PY{l+s+s1}{y}\PY{l+s+s1}{\PYZsq{}}\PY{p}{,} \PY{l+s+s1}{\PYZsq{}}\PY{l+s+s1}{b}\PY{l+s+s1}{\PYZsq{}}\PY{p}{,} \PY{l+s+s1}{\PYZsq{}}\PY{l+s+s1}{pink}\PY{l+s+s1}{\PYZsq{}}\PY{p}{,} \PY{l+s+s1}{\PYZsq{}}\PY{l+s+s1}{purple}\PY{l+s+s1}{\PYZsq{}}\PY{p}{]}
    
    \PY{k}{for} \PY{n}{i}\PY{p}{,} \PY{n}{e} \PY{o+ow}{in} \PY{n+nb}{enumerate}\PY{p}{(}\PY{n}{err}\PY{p}{)}\PY{p}{:}
        \PY{n}{fmin}\PY{p}{,} \PY{n}{fmax}\PY{p}{,} \PY{n}{\PYZus{}} \PY{o}{=} \PY{n}{IC}\PY{p}{(}\PY{n}{mean}\PY{p}{,} \PY{n}{std}\PY{p}{,} \PY{n}{e}\PY{p}{)}
        \PY{n}{xa}  \PY{o}{=} \PY{n}{np}\PY{o}{.}\PY{n}{linspace}\PY{p}{(}\PY{n}{limmin}\PY{p}{,} \PY{n}{fmin}\PY{p}{,} \PY{l+m+mi}{100}\PY{p}{)}
        \PY{n}{xb}  \PY{o}{=} \PY{n}{np}\PY{o}{.}\PY{n}{linspace}\PY{p}{(}\PY{n}{fmax}\PY{p}{,} \PY{n}{limmax}\PY{p}{,} \PY{l+m+mi}{100}\PY{p}{)}
        
        \PY{n}{y1a} \PY{o}{=} \PY{n}{np}\PY{o}{.}\PY{n}{zeros}\PY{p}{(}\PY{n}{xa}\PY{o}{.}\PY{n}{shape}\PY{p}{)}
        \PY{n}{y1b} \PY{o}{=} \PY{n}{np}\PY{o}{.}\PY{n}{zeros}\PY{p}{(}\PY{n}{xb}\PY{o}{.}\PY{n}{shape}\PY{p}{)}
        
        \PY{n}{y2a} \PY{o}{=} \PY{n}{st}\PY{o}{.}\PY{n}{norm}\PY{o}{.}\PY{n}{pdf}\PY{p}{(}\PY{n}{xa}\PY{p}{,} \PY{n}{mean}\PY{p}{,} \PY{n}{std}\PY{p}{)}
        \PY{n}{y2b} \PY{o}{=} \PY{n}{st}\PY{o}{.}\PY{n}{norm}\PY{o}{.}\PY{n}{pdf}\PY{p}{(}\PY{n}{xb}\PY{p}{,} \PY{n}{mean}\PY{p}{,} \PY{n}{std}\PY{p}{)}
        
        \PY{n}{pl}\PY{o}{.}\PY{n}{fill\PYZus{}between}\PY{p}{(}\PY{n}{xa}\PY{p}{,} \PY{n}{y1a}\PY{p}{,} \PY{n}{y2a}\PY{p}{,} \PY{n}{color}\PY{o}{=}\PY{n}{colors}\PY{p}{[}\PY{n}{i}\PY{p}{]}\PY{p}{,} \PY{n}{alpha}\PY{o}{=}\PY{l+m+mf}{0.2}\PY{p}{,} \PY{n}{label}\PY{o}{=}\PY{l+s+sa}{f}\PY{l+s+s1}{\PYZsq{}}\PY{l+s+s1}{Conf. de }\PY{l+s+si}{\PYZob{}}\PY{l+m+mi}{1}\PY{o}{\PYZhy{}}\PY{n}{ea}\PY{l+s+si}{:}\PY{l+s+s1}{.2\PYZpc{}}\PY{l+s+si}{\PYZcb{}}\PY{l+s+s1}{ à }\PY{l+s+si}{\PYZob{}}\PY{l+m+mi}{1}\PY{o}{\PYZhy{}}\PY{n}{e}\PY{l+s+si}{:}\PY{l+s+s1}{.2\PYZpc{}}\PY{l+s+si}{\PYZcb{}}\PY{l+s+s1}{\PYZsq{}}\PY{p}{)}
        \PY{n}{pl}\PY{o}{.}\PY{n}{fill\PYZus{}between}\PY{p}{(}\PY{n}{xb}\PY{p}{,} \PY{n}{y1b}\PY{p}{,} \PY{n}{y2b}\PY{p}{,} \PY{n}{color}\PY{o}{=}\PY{n}{colors}\PY{p}{[}\PY{n}{i}\PY{p}{]}\PY{p}{,} \PY{n}{alpha}\PY{o}{=}\PY{l+m+mf}{0.2}\PY{p}{,} \PY{n}{label}\PY{o}{=}\PY{k+kc}{None}\PY{p}{)}
        
        \PY{n}{limmin} \PY{o}{=} \PY{n}{fmin}
        \PY{n}{limmax} \PY{o}{=} \PY{n}{fmax}
        \PY{n}{ea}     \PY{o}{=} \PY{n}{e}
        

\PY{n}{figHists} \PY{o}{=} \PY{n}{plt}\PY{o}{.}\PY{n}{figure}\PY{p}{(}\PY{l+m+mi}{4}\PY{p}{,} \PY{n}{figsize}\PY{o}{=}\PY{p}{(}\PY{l+m+mi}{10}\PY{p}{,}\PY{l+m+mi}{3}\PY{p}{)}\PY{p}{)}

\PY{n}{pl} \PY{o}{=} \PY{n}{plt}\PY{o}{.}\PY{n}{subplot}\PY{p}{(}\PY{l+m+mi}{1}\PY{p}{,} \PY{l+m+mi}{2}\PY{p}{,} \PY{l+m+mi}{1}\PY{p}{)}
\PY{n}{plt}\PY{o}{.}\PY{n}{title}\PY{p}{(}\PY{l+s+s1}{\PYZsq{}}\PY{l+s+s1}{Primeira Frequência}\PY{l+s+s1}{\PYZsq{}}\PY{p}{)}

\PY{n}{plt}\PY{o}{.}\PY{n}{plot}\PY{p}{(}\PY{n}{x1}\PY{p}{,} \PY{n}{rv\PYZus{}f1}\PY{o}{.}\PY{n}{pdf}\PY{p}{(}\PY{n}{x1}\PY{p}{)}\PY{p}{,} \PY{n}{label}\PY{o}{=}\PY{l+s+s1}{\PYZsq{}}\PY{l+s+s1}{FDP}\PY{l+s+s1}{\PYZsq{}}\PY{p}{)}
\PY{n}{plotIC}\PY{p}{(}\PY{n}{mean\PYZus{}f1}\PY{p}{,} \PY{n}{std\PYZus{}f1}\PY{p}{,} \PY{n}{err}\PY{o}{=}\PY{p}{[}\PY{l+m+mf}{0.025}\PY{p}{,} \PY{l+m+mf}{0.050}\PY{p}{,} \PY{l+m+mf}{0.100}\PY{p}{,} \PY{l+m+mf}{0.5}\PY{p}{,} \PY{l+m+mi}{1}\PY{p}{]}\PY{p}{)}

\PY{n}{plt}\PY{o}{.}\PY{n}{grid}\PY{p}{(}\PY{k+kc}{False}\PY{p}{)}
\PY{n}{plt}\PY{o}{.}\PY{n}{ylabel}\PY{p}{(}\PY{k+kc}{None}\PY{p}{)}
\PY{n}{plt}\PY{o}{.}\PY{n}{xlabel}\PY{p}{(}\PY{l+s+s1}{\PYZsq{}}\PY{l+s+s1}{Frquência em Hz}\PY{l+s+s1}{\PYZsq{}}\PY{p}{)}
\PY{n}{plt}\PY{o}{.}\PY{n}{xlim}\PY{p}{(}\PY{n}{x1}\PY{o}{.}\PY{n}{min}\PY{p}{(}\PY{p}{)}\PY{p}{,} \PY{n}{x1}\PY{o}{.}\PY{n}{max}\PY{p}{(}\PY{p}{)}\PY{p}{)}
\PY{n}{plt}\PY{o}{.}\PY{n}{ylim}\PY{p}{(}\PY{l+m+mi}{0}\PY{p}{)}
\PY{n}{plt}\PY{o}{.}\PY{n}{legend}\PY{p}{(}\PY{n}{ncol}\PY{o}{=}\PY{l+m+mi}{3}\PY{p}{,} \PY{n}{loc}\PY{o}{=}\PY{l+s+s1}{\PYZsq{}}\PY{l+s+s1}{upper center}\PY{l+s+s1}{\PYZsq{}}\PY{p}{,} \PY{n}{bbox\PYZus{}to\PYZus{}anchor}\PY{o}{=}\PY{p}{(}\PY{l+m+mf}{1.1}\PY{p}{,}\PY{o}{\PYZhy{}}\PY{l+m+mf}{0.2}\PY{p}{)}\PY{p}{,} \PY{n}{fontsize}\PY{o}{=}\PY{l+m+mi}{9}\PY{p}{,} \PY{n}{frameon}\PY{o}{=}\PY{k+kc}{False}\PY{p}{)}


\PY{n}{pl} \PY{o}{=} \PY{n}{plt}\PY{o}{.}\PY{n}{subplot}\PY{p}{(}\PY{l+m+mi}{1}\PY{p}{,} \PY{l+m+mi}{2}\PY{p}{,} \PY{l+m+mi}{2}\PY{p}{)}
\PY{n}{plt}\PY{o}{.}\PY{n}{title}\PY{p}{(}\PY{l+s+s1}{\PYZsq{}}\PY{l+s+s1}{Segunda Frequência}\PY{l+s+s1}{\PYZsq{}}\PY{p}{)}

\PY{n}{plt}\PY{o}{.}\PY{n}{plot}\PY{p}{(}\PY{n}{x2}\PY{p}{,} \PY{n}{rv\PYZus{}f2}\PY{o}{.}\PY{n}{pdf}\PY{p}{(}\PY{n}{x2}\PY{p}{)}\PY{p}{,} \PY{n}{label}\PY{o}{=}\PY{l+s+s1}{\PYZsq{}}\PY{l+s+s1}{FDP}\PY{l+s+s1}{\PYZsq{}}\PY{p}{)}
\PY{n}{plotIC}\PY{p}{(}\PY{n}{mean\PYZus{}f2}\PY{p}{,} \PY{n}{std\PYZus{}f2}\PY{p}{,} \PY{n}{err}\PY{o}{=}\PY{p}{[}\PY{l+m+mf}{0.025}\PY{p}{,} \PY{l+m+mf}{0.050}\PY{p}{,} \PY{l+m+mf}{0.100}\PY{p}{,} \PY{l+m+mf}{0.5}\PY{p}{,} \PY{l+m+mi}{1}\PY{p}{]}\PY{p}{)}

\PY{n}{plt}\PY{o}{.}\PY{n}{grid}\PY{p}{(}\PY{k+kc}{False}\PY{p}{)}
\PY{n}{plt}\PY{o}{.}\PY{n}{ylabel}\PY{p}{(}\PY{k+kc}{None}\PY{p}{)}
\PY{n}{plt}\PY{o}{.}\PY{n}{xlabel}\PY{p}{(}\PY{l+s+s1}{\PYZsq{}}\PY{l+s+s1}{Frquência em Hz}\PY{l+s+s1}{\PYZsq{}}\PY{p}{)}
\PY{n}{plt}\PY{o}{.}\PY{n}{xlim}\PY{p}{(}\PY{n}{x2}\PY{o}{.}\PY{n}{min}\PY{p}{(}\PY{p}{)}\PY{p}{,} \PY{n}{x2}\PY{o}{.}\PY{n}{max}\PY{p}{(}\PY{p}{)}\PY{p}{)}
\PY{n}{plt}\PY{o}{.}\PY{n}{ylim}\PY{p}{(}\PY{l+m+mi}{0}\PY{p}{)}
\end{Verbatim}
\end{tcolorbox}

            \begin{tcolorbox}[breakable, size=fbox, boxrule=.5pt, pad at break*=1mm, opacityfill=0]
\prompt{Out}{outcolor}{14}{\boxspacing}
\begin{Verbatim}[commandchars=\\\{\}]
(0.0, 0.05567011217681224)
\end{Verbatim}
\end{tcolorbox}
        
    \begin{center}
    \adjustimage{max size={0.9\linewidth}{0.9\paperheight}}{trab3a_files/trab3a_27_1.pdf}
    \end{center}
    { \hspace*{\fill} \\}
    
    \hypertarget{considerauxe7uxf5es-finais}{%
\section{Considerações finais}\label{considerauxe7uxf5es-finais}}

A consideração das incertezas intrinsecas dos materiais usados na
construção do modelo reduzido tornou seu comportamento também incerto.
As incertezas existentes nas propriedades dos materiais foram propagadas
no modelo através da realização de \(N=5\times10^6\) simulações de Monte
Carlo e, posteriormente, as \(N\) respostas observadas foram
caracterizadas através de análises estatísticas. Observou-se que ambas
as frequências naturais de vibração do modelo reduzido são bem descritas
por variáveis de distribuição normal/Gaussiana:

\[ f_1 ~\sim ~N(\mu_{f1}=18.67 Hz; \;\; \sigma_{f1}=1.21 Hz)\]

\[ f_2 ~\sim ~N(\mu_{f2}=116.16 Hz; \; \sigma_{f2}=7.52 Hz)\]

onde as médias das variáveis aleatórias são aproximadamente as
frequências obtidas com as propriedades médias dos materiais.

    Conhecidas as distribuições de probabilidade das frequências naturais
foram determinados seus intervalos de confiança, onde, com uma confiança
de 95\%, tem-se que as frequências são:

\[16.30 Hz \le f_1 \le 21.03 Hz\]

\[101.41 Hz \le f_2 \le 130.91 Hz\]


    % Add a bibliography block to the postdoc
    
    
    
\end{document}
