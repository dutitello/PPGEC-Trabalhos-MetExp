C:/Programacao/_MinhasBibliotecas/Jupyter2Latex/latex-template/article-base.tex

\author{Eduardo Pagnussat Titello}
\title{PEC00144 - Métodos Experimentais em Engenharia Civil}
\subtitle{Prévia do trabalho final}
\date{Janeiro de 2021}


\begin{document}

\maketitle

\section{Introdução}
Este trabalho tem por objetivo introduzir o modelo reduzido que será construído e estudado na disciplina. Serão apresentados o conceito do modelo, a tabela de escalas adotadas e o tipo de grandeza medida experimentalmente.


A estrutura a ser representada pelo modelo reduzido é um edifício hipotético, com planta quadrada de dimensões $10 \times 10 \, m$, contendo 5 lajes acima do nível do solo e pé direito de $3 \, m$. A estrutura é formada por 9 pilares de $25 \times 25 \, cm$,  além das vigas e lajes. Esse é apresentado na figura \ref{fig:edfestudo}.


\begin{figure}
	\centering
	\caption{Edifício de estudo}
	\includegraphics[scale=0.9]{../Images/VistaeCorte}
	\label{fig:edfestudo}
\end{figure}


O modelo reduzido a








\pagebreak \part{Antigo}

\textbf{Nesta primeira etapa, elabore a ideia do modelo reduzido que será utilizado no restante da disciplina. Apresente a tabela de escalas e estabeleça que tipo de grandeza será medida no experimento.}

\section{Introdução}

A análise dinâmica de edifícios de múltiplos pavimentos pode ser realizada matematicamente com modelos de diferentes graus de refinamento. Esses modelos costumam ter suas massas concentradas nos pavimentos e podem ser do tipo \textit{shear building} ou tridimensionais (Soriano, 2014).

O modelo \textit{shear building}, clássico e de grande simplicidade, supõe pisos indeformáveis e colunas inextensíveis. Esse modelo equivale à uma coluna de trechos de rigidezes iguais à soma das rigidezes à flexão dos pilares de cada pavimento da edificação (Soriano, 2014). Na figura \ref{fig:shearb} é apresentado um modelo de edifício com duas colunas de pilares e sua representação como \textit{shear building}.


\begin{figure}
	\centering
	\caption{Modelo de edifício e sua representação como \textit{shear building}.\\ \small{(Adaptado de Soriano, 2014)}}
	\includegraphics[scale=0.39]{../Images/ShearBuilding}
	\label{fig:shearb}
\end{figure}


\section{Estrutura estudada}

O edifício apresentado na \ref{fig:edfestudo} 





\section{Modelo reduzido}



Adotando um edifício retangular de planta 

Embora o modelo matemático do \textit{shear building} seja formado por uma única coluna, conforme representação à direita na figura \ref{fig:shearb}, para fins de experimentação o uso de tal modelo pode ser inviável. Como exemplo tem-se a construção de modelos com amortecedores, que requerem uma área para sua instalação, 

A construção de modelos reduzidos com amortecedores, por exemplo, requer um modelo bidimensional, 

Dada a necessidade de espaço e estabilidade para instalação de sensores, massas para ajuste de escala, 



\section{Metodologia}
Aaaa


\section{Materiais}
Aaaa


\section{Fatores de escala}
Aaaa


\section{Resultados esperados}
Aaaa


\section{Referências Bibliográficas}
Aaaaa
    
\end{document}


